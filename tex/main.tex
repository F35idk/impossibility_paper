\documentclass{llncs}
\pagestyle{plain}

\usepackage{src/mypreamble}

\NewDocumentCommand{\gameG}{o}{%
  \IfValueTF{#1}{%
    \ensuremath{G^{*}}\xspace%
  }{%
    \ensuremath{G}\xspace%
  }%
}
\NewDocumentCommand{\redR}{o}{%
  \IfValueTF{#1}{%
    \ensuremath{\mathcal{R}_{#1}}\xspace%
  }{%
    \ensuremath{\mathcal{R}}\xspace%
  }%
}
\newcommand{\redM}{\ensuremath{\mathcal{M}}\xspace}
\newcommand{\ICA}{\ensuremath{\mathsf{ICA}}\xspace}
\newcommand{\IMP}{\ensuremath{\mathsf{IMP(\prot)}}\xspace}
\newcommand{\KEM}{\ensuremath{\mathcal{E}}\xspace}
\newcommand{\OwFS}{\ensuremath{\mathsf{OwFS}}\xspace}
\newcommand{\wFS}{\ensuremath{\mathsf{wFS}}\xspace}
\newcommand{\KEMprot}{\ensuremath{\prot(\KEM)}\xspace}
\newcommand{\epsV}{\ensuremath{\epsilon_{V}}\xspace}
\newcommand{\qeq}{\ensuremath{\overset{?}{=}}\xspace}
\newcommand{\qle}{\ensuremath{\overset{?}{<}}\xspace}
\newcommand{\proc}[2]{\procedure[jot=-0.5mm,codesize=\normalsize,linenumbering]{#1}{#2}}
\newcommand{\protpc}[1]{\pseudocode[]{#1}}
\newcommand{\protpcsmall}[1]{\pseudocode[jot=-0.5mm,codesize=\scriptsize]{#1}}
\newcommand{\vect}[1]{\ensuremath{\mathbf{#1}}\xspace}
\newcommand{\eventE}{\ensuremath{\mathsf{E}}\xspace}
\newcommand{\bigTheta}{\ensuremath{\Theta}}

\newcommand{\Simple}{\ensuremath{\mathcal{S}}\xspace}
\newcommand{\OWECPA}{\ensuremath{\mathsf{OW}\text{-}\mathsf{ECPA}}\xspace}
\newcommand{\INDECPA}{\ensuremath{\mathsf{IND}\text{-}\mathsf{ECPA}}\xspace}
\newcommand{\ECPA}{\ensuremath{\mathsf{ECPA}}\xspace}
\newcommand{\Enc}{\ensuremath{\mathsf{Enc}}\xspace}
\newcommand{\Setup}{\ensuremath{\mathsf{Setup}}\xspace}
\newcommand{\KG}{\ensuremath{\mathsf{KG}}\xspace}
\newcommand{\Encaps}{\ensuremath{\mathsf{E}}\xspace}
\newcommand{\Decaps}{\ensuremath{\mathsf{D}}\xspace}
\newcommand{\Reveal}{\ensuremath{\mathsf{Reveal}}\xspace}
\newcommand{\RevState}{\ensuremath{\mathsf{RevState}}\xspace}
\newcommand{\Corrupt}{\ensuremath{\mathsf{Corrupt}}\xspace}
\newcommand{\params}{\ensuremath{\mathsf{par}}\xspace}
\newcommand\given[1][]{\:#1\vert\:}

\newcommand{\Init}{\ensuremath{\mathsf{Init}}\xspace}
\newcommand{\Run}{\ensuremath{\mathsf{Run}}\xspace}
\newcommand{\Respond}{\ensuremath{\mathsf{Respond}}\xspace}
\newcommand{\Win}{\ensuremath{\mathsf{Win}}\xspace}

\newcommand{\setC}{\ensuremath{\mathcal{C}}\xspace}
\newcommand{\setM}{\ensuremath{\mathcal{M}}\xspace}
\newcommand{\setT}{\ensuremath{\mathcal{T}}\xspace}
\newcommand{\setSK}{\ensuremath{\mathcal{SK}}\xspace}
\newcommand{\setST}{\ensuremath{\mathcal{ST}}\xspace}
\newcommand{\Round}{\ensuremath{\mathsf{Round}}\xspace}
\newcommand{\eventi}[1]{\ensuremath{\mathsf{E}_{#1}}\xspace}

\newcommand{\GenAKE}{\ensuremath{\mathsf{Gen}_{\prot}}\xspace}
\newcommand{\InitI}{\ensuremath{\mathsf{Init_{I}}}\xspace}
\newcommand{\DerR}{\ensuremath{\mathsf{Der_{R}}}\xspace}
\newcommand{\DerI}{\ensuremath{\mathsf{Der_{I}}}\xspace}
\newcommand{\OwFSst}{\ensuremath{\mathsf{OwFS}\text{-}\mathsf{St}}\xspace}
\newcommand{\INDwFSst}{\ensuremath{\mathsf{IND}\text{-}\mathsf{wFS}\text{-}\mathsf{St}}\xspace}
\newcommand{\sID}{\ensuremath{\mathsf{sID}}\xspace}
\newcommand{\redL}{\ensuremath{\mathcal{L}}\xspace}

\newcommand{\List}{\ensuremath{\mathsf{List}}\xspace}

%%% Local Variables:
%%% mode: latex
%%% TeX-master: "main"
%%% End:


\title{Strong Impossibility on Tightly Secure Key Encapsulation and Key Exchange}
%\subtitle{Subtitle}
\titlerunning{Strong Impossibility on Tightly Secure Key Exchange}
\date{\now}
%\author{Authors}
%\institute{Institutes}



\begin{document}

\maketitle

\begin{abstract}
	Authenticated key exchange (AKE) protocols run at a large scale, and their security loss affects their security guarantee and the theoretically sound parameter selection. Using key encapsulation mechanisms (KEMs) is a well-known approach in constructing AKE protocols, for instance, the lattice-based ones. In this paper, we study tightness of KEM-based AKE protocols with a focus on lattice-based constructions.
	
	\quad More precisely, we consider the enhanced chosen-plaintext (ECPA) security of KEM, a notion proposed by Han, Liu, and Gu (ASIACRYPT 2021). We use a combinatorial argument to show that if an adversary involves $2 \lambda$ many users and $11 \lambda p(\lambda)$ ciphertexts, then the security reduction has to lose a linear factor on the number of users $\mu$ in the \emph{standard model}. Hence, its security cannot be tight. In the previous impossibility result of Han et al., it requires the KEM to have polynomially-bounded ranks and it only rules out a class of KEMs based on the Diffie-Hellman assumption. However, our result is more general and does not require polynomially-bounded ranks, and thus it can capture those KEMs based on lattices. Since the ECPA-secure KEM is related to the AKE security, an application of our result is excluding any tightly secure AKE, where the protocol transcripts are independent of the long-term secret keys.
\end{abstract}

%\blankpage


\section{Introduction}
Key encapsulation mechanisms (KEMs) and authenticated key exchange (AKE) protocols are essential primitives in public-key cryptography.
In short, a KEM allows a user to encapuslate a random key $K$ by outputting a ciphertext $C$ using the public key of another user. 
%a pair \((k,c)\)
%consisting of a random key \(k\) and a ciphertext \(c\),
With the corresponding secret key, the designated user can decrypt the ciphertext \(C\) and recover the same key $K$.
%The ciphertext \(c\) is referred to as an \emph{encapsulation} of the key \(k\),
%under the public key of the designated user.
%Any public key encryption scheme easily yields a KEM, and the opposite implication is also true (cite).
KEMs are used as building blocks for many complex public-key cryptosystems. Among them, AKE protocols are (arguably) the most important application, since AKE protocols can exchange a session key between two users to establish a secure channel.
For instance, the promising post-quantum TLS replacement KEMTLS \cite{CCS:SchSteWig20} is a well-known example of KEM-based AKE.


\heading{Security of KEM-based AKE.}
The security requirement for AKE is that an adversary $\advA$ cannot learn information about the shared session key even if $\advA$ manipulates the protocol messages, adaptively corrupt some user long-term keys, and reveal keys of some sessions. This is more complicated than 

\iffalse
\begin{figure}[t]
	\centering
	\begin{pcimage}
		\fbox{  
			\centering
			\pseudocode[colspace=-2.5em]{%
				\textbf{Party } P_i: (\akeCCApk_i, \akeCCAsk_i) \<  \< \textbf{Party } P_j: (\akeCCApk_j, \akeCCAsk_j) \\[0.1\baselineskip][\hline]
				\<\< \\[-0.8\baselineskip]
				(\akeCPApk, \akeCPAsk) \from \akeCPAKEMKG(1^\secpar)\\[-0.2\baselineskip] 
				(\akeCCAct_j, \akeCCAkey_j) \from \akeCCAKEMEnc(\akeCCApk_j) \\[-1.2\baselineskip]
				%\akeState := (\akeCPApk, \akeCPAsk, \akeCCAct_j, \akeCCAkey_j)\\
				\< \sendmessage{->}{top={$(\akeCPApk,\akeCCAct_j)$}, length=3cm} \< \\[-1.2\baselineskip]
				\<\< (\akeCPAct, \akeCPAkey) \from \akeCPAKEMEnc(\akeCPApk)\\
				\<\< \akeCCAkey_j := \akeCCAKEMDec(\akeCCAsk_j, \akeCCAct_j)\\
				\<\< (\akeCCAct_i, \akeCCAkey_i) \from \akeCCAKEMEnc(\akeCCApk_i)\\[-1.2\baselineskip]
				\< \sendmessageleft{top={$(\akeCPAct, \akeCCAct_i)$}, length=3cm} \< \\[-0.8\baselineskip]
				\akeCPAkey := \akeCPAKEMDec(\akeCPAsk, \akeCPAct) \<\< \context := (\akeCCApk_i, \akeCCApk_j, \akeCPApk, \akeCCAct_i, \akeCCAct_j, \akeCPAct) \\[-0.2\baselineskip] 
				\akeCCAkey_i := \akeCCAKEMDec(\akeCCAsk_i, \akeCCAct_i) \<\< 
				\SessionKey := \akeHash(\context, \akeCCAkey'_i, \akeCCAkey'_j, \akeCPAkey') \\[-0.2\baselineskip] 
				\context := (\akeCCApk_i, \akeCCApk_j, \akeCPApk, \akeCCAct_i, \akeCCAct_j, \akeCPAct)\\[-0.2\baselineskip] 
				%				\<\< \\[-0.2\baselineskip] 
				\SessionKey := \akeHash(\context, \akeCCAkey'_i, \akeCCAkey'_j, \akeCPAkey') \\[0.1\baselineskip][\hline]
				\<\< \\[-0.8\baselineskip]
				%				\gamechange{$\SessionKey := \akeHash(\context, \akeCCAkey'_i, \akeCCAkey'_j, \akeCPAkey')$}	
				\< \hfil \akeCCAkey'_i := \akeCCAkey_i, \gamechange{$\akeCCAkey'_i:= \akeNCHash{}{(\akeCCApk_i, \akeCCAct_i, \akeCCAkey_i)}$}  \\
				\< \akeCCAkey'_j := \akeCCAkey_j, \gamechange{$\akeCCAkey'_j:= \akeNCHash{}{(\akeCCApk_j, \akeCCAct_j, \akeCCAkey_j)}$}\\
				\<\akeCPAkey' := \akeCPAkey, \gamechange{$\akeCPAkey':= \akeNCHash{}{(\akeCPApk, \akeCPAct, \akeCPAkey)}$} 
			}
		}
		%		\pcdraw{
			%			\path[->] ([xshift=1.5cm,yshift=-1.2mm] start.south) edge node[right]{$\ \akeState$} ([xshift=1.5cm, yshift=3mm] end.north);
			%		}
	\end{pcimage}
	\caption{Our two approaches of constructing tightly secure AKE protocols between two parties from  secure KEMs, $\akeCCAKEM = (\akeCCAKEMSetup, \akeCCAKEMKG, \akeCCAKEMEnc, \akeCCAKEMDec)$ and $\akeCPAKEM = (\akeCPAKEMSetup, \akeCPAKEMKG, \akeCPAKEMEnc, \akeCPAKEMDec)$. Our two approaches only differ on how the final session keys are derived. We mark the difference in our second approach with \gamechange{gray}. $\akeHash$ and $\akeNCHash{}$ are two independent hash functions.}
	%	\jpnote{Need lots of adaptations}
	\label{fig:scheme:vis-akescheme}
\end{figure}
\fi


\heading{Security of KEM.}
The standard notions of security for a key exchange mechanism are so-called IND-CPA and IND-CCA security.
These notions capture resistance to attacks in which an adversary receives a single
encapsulation-key-pair \((k,c)\) where \(c\) encapsulates \(k\),
and must distinguish the key \(k\) from a truly random key.
IND-CPA and IND-CCA security blablabla.
On the other hand, in a real-world setting,
an adversary may not be limited to seeing a single encapsulation \(c\)
--- it could receive several encapsulations and then adaptively select which one to attack.
Additionally, the adversary may see encapsulations under different public keys belonging to different users.
In certain contexts, the imagined adversary could even have \emph{inside information}
on the keys encapsulated by certain ciphertexts, or on the secret keys of certain users.
As an example, if the key exchange mechanism is used as a component in a key exchange protocol,
then there could be a massive number of users and sessions executing the protocol in parallell,
and hence making use of the KEM.
In such a context, it is not unreasonable to expect that an adversary blabla.
To model these adversarial capabilities,
we can extend the basic IND-CPA and IND-CCA security notions.
blabla.
The basic form of such an extended security notion is ECPA security (ref.?),
which is defined in terms of an adversary that may receive multiple encapsulations
on several different users' public keys, and adaptively reveal encapsulations and corrupt users.


\heading{Tight AKE Security from KEM.}
Although ECPA security appears stronger than the standard IND-CPA notion,
it is an easy exercise to prove that any IND-CPA secure KEM is also ECPA secure.
As is standard in provably secure cryptography, this implication is proven via a so-called \emph{reduction}.
In short, a reduction is an algorithm which turns any attack on a cryptographic scheme
into an attack on an underlying problem or scheme.
Using reductions,
cryptographers can provably relate the security of different cryptographic constructions,
and moreover design cryptographic schemes
whose security can be reduced to the difficulty of simpler,
presumed-to-be-difficult computational problems.
Although we can relate the IND-CPA and ECPA security of a KEM using a reduction,
the issue with the reduction in question is that it incurs a potentially large loss
in the quality of the security guarantee. To be more precise,
if the success probability blabla.
\[
  \epsilon_{\redR} \ge ??
\]

What we would much rather prefer is if the blabla
\[
  \epsilon_{\redR} \ge ??
\]
In this case, we call the reduction \emph{tight}.
Tight security reductions are desirable from a practical point of view,
since they give better quantitative security guarantees on the cryptographic construction in question.
This allows the cryptographic construction to be instantiated with smaller parameters, for a given level of security,
leading to increased efficiency.
It is also \TODO{blabla no asymptotic degradation in the security guarantee, theoretically interesting}.
There has been extensive research on constructing cryptographic schemes with tight security reductions (cite).
Blabla.

Unfortunately,
for many realistic, ``extended'' security notions which handle multiple users with adaptive reveals and corruptions,
constructing cryptographic primitives with tight security reductions appears to be a difficult problem.
Blabla \TODO{impossibility results blabla}.
We are thus interested in the following question: is it possible for a KEM to have a tight reduction
which proves ECPA security, from any standard cryptographic assumption?

\subsubsection{HLG21's impossibility result.}

As HLG21 observe, blabla.
and yet no tight reductions are known for these KEMs.

Unlike blabla, HLG21's result blabla.
As HLG21 show, many well-known KEMs satisfy the polynomial rank requirement, such as blablabla.

One important class of KEM constructions which the result of HLG21 does not cover is lattice-based KEMs.
Indeed, as we show in ????, blabla exponential rank.
This leaves open the question of whether a tight security reduction exists e.g for the Regev KEM,
or for other lattice-based constructions.

In this paper, we rule out the existence of such a reduction.
That is, we show the following result.

\TODO{state our results informally.}

\TODO{Also discuss the applications to AKE. Emphasize the ``unconditional'' nature of the results, i.e that they make essentially no requirements on the primitives. Mention how our AKE results answer an open question from JKRS21 about the existence of tight AKE with state reveals in the standard model.}


% In provably secure cryptography, the security of a cryptographic construction is proven via a so-called \emph{reduction}.
% Using a reduction, any attack on a given cryptographic primitive is
% turned into an attack on an underlying, presumed-to-be difficult problem.
% This way, the security of the primitive is reduced to the difficulty of the aforementioned computational problem.
% We can measure the quality of a reduction by considering its \emph{tightness}, which blabla.



%%% Local Variables:
%%% mode: latex
%%% TeX-master: "../main"
%%% End:



\section{Proof outline}


%%% Local Variables:
%%% mode: latex
%%% TeX-master: "../main"
%%% End:



\section{Preliminaries}

\subsection{Key exchange mechanisms}

\TODO{Add definition of a KEM.}

\begin{definition}[\((1-\delta)\)-correctness of a KEM]\label{def:corr}
  Let \(\KEM = (\Setup, \KG, \Encaps, \Decaps)\) be a KEM
  and let \(\delta : \mathbb{N} \to \mathbb{R}\) be a function of the security parameter.
  Then \KEM is said to have \emph{\((1-\delta)\)-correctness} if
  \begin{equation}
    \prob{\Decaps{}(\sk,c) \ne k : \params \assign \Setup{}(1^{\lambda}), (\sk,\pk) \assign \KG{}(\params), (c,k) \assign \Encaps{}(\pk)} \le \delta (\lambda)
  \end{equation}
\end{definition}


\begin{definition}[OW-ECPA security]\label{def:OWECPA_sec}
  Let \(\KEM = (\Setup, \KG, \Encaps, \Decaps)\) be a KEM
  and let \advA{} be an adversary in the \OWECPA{} game against \KEM (see Figure \ref{fig:ecpa}).
  Then the advantage of \advA{} against \OWECPA in \KEM is given by
  \begin{equation}
    \abs*{\prob{\OWECPA{}_{\KEM}^{\advA}(\mu,\rho,\lambda) \implies 1} - 1/\abs{\keyspace}},
  \end{equation}
  where \(\keyspace = \keyspace{}(\params)\) is the encapsulation key space of \KEM.
  \TODO{advantage notation?}
\end{definition}

\begin{figure}
  \begin{pcvstack}[center,space=0.5cm]
    \proc{Game $\OWECPA_{\KEM}^{\advA}(\mu, \rho, \lambda)$}{
      L_{rev}, L_{corr} \assign \List.\\
      \params \assign \KEM{}.\Setup{}(1^{\lambda}).\\
      \pcfor i \in [\mu]:\\
      \t (\sk_i,\pk_i) \assign \KG(\params).\\
      \oracle := \{\Enc, \Reveal, \Corrupt\}.\\
      (i^*, j^*, k^*) \assign \advA^{\oracle}(\params,\pk_1, \ldots, \pk_{\mu}).\\
      \pcif i^{*} \notin L_{corr} \land i^{*} \notin L_{rev} \land k^{*} = k_{i^{*},j^{*}}:\\
      \t \pcreturn 1.\\
      \pcelse:\\
      \t \pcreturn 0.
    }
    \proc{Oracle \Enc{}$(i)$}{
      \text{Assume this is the \(j\)-th query to \Enc{} on user \(i\) with } j \in [\rho].\\
      (c_{i,j}, k_{i,j}) \assign \Encaps{}(\pk_i).\\
      \pcreturn c_{i,j}.
    }
    \begin{pchstack}[space=1cm]
      \proc{Oracle \Reveal{}$(i,j)$}{
        L_{rev} \assign L_{rev} \cup \{(i,j)\}.\\
        \pcreturn k_{i,j}.
      }
      \proc{Oracle \Corrupt{}$(i)$}{
        L_{corr} \assign L_{corr} \cup \{i\}.\\
        \pcreturn \sk_i.
      }
    \end{pchstack}
  \end{pcvstack}\caption{}\label{fig:ecpa}
\end{figure}

\TODO{define simple reductions.}

\subsection{Computational assumptions}

When we prove our tightness impossibility results,
we will only be able to rule out tight reductions from a specific class of computational assumptions.
That is, we can only show that a tight reduction to a certain security game \gameG is impossible
if the assumption being reduced from is sufficiently restricted.
This is in some sense an inherent limitation, as we could not rule out a tight reduction to the game \gameG
from \emph{any} computational assumption
(a standard counterexample to this is to use the assumption that the game \gameG is secure.
Clearly there is a tight reduction from this assumption to the game \gameG).
Hence we need the assumption being reduced from to be ``weaker''
than the game \gameG, in a certain sense.

In the impossibility results of \TODO{cite}, the class of computational assumptions
in question are the so-called \emph{non-interactive} assumptions.
These are modeled by experiments in which an adversary receives a challenge \(c\)
and must output a solution \(s\) to the challenge, with no further interaction between the adversary and the experiment.
Following an observation of \TODO{cite}, we will actually slightly expand the class of computational assumptions
from which we can rule out tight reductions, at essentially no cost to the complexity of our proofs.

In \TODO{cite}, it is shown that a tight, simple reduction from the \emph{Strong Diffie-Hellman} assumption
to the weak forward secrecy of a certain class of key exchange protocols cannot exist,
unless the Strong Diffie-Hellman assumption is itself false.
The key point here is that the Strong Diffie-Hellman game is interactive,
and hence does not fit into the class of non-interactive assumptions considered in e.g \TODO{cite}.
Nevertheless \TODO{cite} are able to apply a similar meta-reduction strategy
to the one pioneered in \TODO{cite} to prove their impossibility result.
We observe that the Strong Diffie-Hellman game satisfies the following two criteria,
which allow the proof of \TODO{cite} to go through.
The first is that the interaction in the Strong Diffie-Hellman game is in some sense stateless.
More precisely, the Strong Diffie-Hellman oracle computes its answer to a query
in a way that is independent of the content of any previous queries.
Second, the ``winning condition'' in the Strong Diffie-Hellman game is essentially independent of
the queries made to the Strong Diffie-Hellman oracle.
\TODO{explain why this is the case.}
\TODO{give at least a vague reason for why this should allow the proofs to work.}

We can define a class of interactive computational assumptions which satisfy the above two criteria.
% This will allow us to prove impossibility results for this class.
We will call such assumptions \emph{simple computational assumptions}.

\begin{definition}[Simple computational assumption]\label{def:simple}
  A simple computational assumption is a tuple \((\Init,\Respond,\Win,\kappa(\cdot))\),
  where \(\kappa : \N \to \R\) is a function,
  \(\Init\) is a probabilistic algorithm
  and \(\Respond\) and \(\Win\) are deterministic algorithms.
  For a given simple computational assumption \(\Simple = (\Init,\Respond,\Win,\kappa)\)
  and an algorithm \advA, we define an experiment \(\SICA_{\Simple}^{\advA}\)
  according to Figure \ref{fig:simple}.

  \TODO{Explain in more detail what the different algorithms actually do.}
  Here, \Win{} defines the ``winning condition'' in the game.
  Notice that \Win{} depends only on the initial state \(\state_{\Simple}\)
  of the game and the challenge solution \(s\) output by \advA
  (in particular, it does not depend on the queries made by \advA
  to \oracle during the game).
  Also, whenever \advA queries the oracle \oracle,
  the response is computed deterministically as \(\Respond(\state_{\Simple},m)\).
  The advantage of \advA{} against \Simple is given by
  \TODO{advantage notation.}
  \begin{equation}
    \abs[\big]{\prob{\SICA(\advA,\lambda) \implies 1} - \kappa(\lambda)}.
  \end{equation}
  \TODO{say something about kappa.}
\end{definition}

\begin{figure}
  \begin{pchstack}[center,space=0.5cm]
    \proc{Game $\SICA_{\Simple}^{\advA}(\lambda)$}{
      \label{code:simple:init} (\state_{\Simple}, c) \assign \Init(1^{\lambda}).\\
      \label{code:simple:output} s \assign \advA^{\oracle}(c).\\
      \label{code:simple:win} b' \assign \Win(\state_{\Simple},s).\\
      \pcreturn b'.
    }
    \proc{Oracle \oracle(m)}{
      \pcreturn \Respond(\state_{\Simple},m).
    }
  \end{pchstack}\caption{}\label{fig:simple}
\end{figure}

\TODO{say something about how this extends non-interactive assumptions.}
\TODO{say something about how CCA-type games can be modeled by this. And stDH}.


%%% Local Variables:
%%% mode: latex
%%% TeX-master: "../main"
%%% End:



\section{The KEM impossibility result}

\begin{definition}[Agreeing pairs]\label{def:agree}
  Let \setSK, \setC, \keyspace be sets and
  let \(f : \setSK \times \setC \to \keyspace\) be a function.
  Fix a \(c \in \setC\) and a pair \((s,s') \in \setSK^{2}\).
  Then \(s\) and \(s'\) are said to \emph{agree on \(c\) with respect to \(f\)} if
  \(
    f(s,c) = f(s',c).
  \)
  When \(f\) is understood from context, we simply say that \(s\) and \(s'\) \emph{agree on \(c\)}.
  On the other hand, if
  \(
    f(s,c) \ne f(s',c),
  \)
  then we say that \(s\) and \(s'\) \emph{disagree on \(c\)} (with respect to \(f\)).
\end{definition}

\begin{definition}[Bad pairs]\label{def:bad}
  \TODO{pick a better name than ``bad''.}
  Let \setSK, \setC, \keyspace, \(f : \setSK \times \setC \to \keyspace\) be as before.
  Fix a set \(A \subset \setC\) and a subset \(S \subset A\),
  along with a pair \((s,s') \in \setSK^{2}\) and a real number \(\alpha > 0\).
  Then \((s,s')\) is said to be \emph{\(\alpha\)-bad for the subset \(S\)} (with respect to \(A\) and \(f\)) if
  \(s\) and \(s'\) agree on every \(c \in A \setminus S\), yet disagree on more than an \(\alpha\)-fraction of \(c \in S\).
  In other words, \((s,s')\) is \(\alpha\)-bad for the subset \(S\) if we have
  \(
    f(s,c) = f(s',c) \text{ for all } c \in A \setminus S,
  \)
  but
  \(
    f(s,c) \ne f(s',c) \text{ for more than } \alpha t \text{ distinct } c \in S.
  \)
  % When \(\alpha\) is clear from context, we simply say that \((s,s')\) is \emph{bad for the subset \(S\)} (with respect to \(A\) and \(f\)).
\end{definition}


\begin{lemma}[Bounding bad pairs]\label{lemma:equiv}
  Let \setSK, \setC, \keyspace, \(f : \setSK \times \setC \to \keyspace\) be as before.
  Let \(d,\rho,t \in \N, \alpha > 0\) with \(t \le \rho/3\)
  and assume that \(\setSK \subset \{0,1\}^{d}\).
  Fix a set \(A \subset \setC\) of size \(\rho\)
  and let \(S \subset A\) be a uniformly random size-\(t\) subset of \(A\).
  Then
  \begin{align}\label{ineq:equiv}
    \prob*{\exists (s,s') \in \setSK^{2} : (s,s') \text{ is \(\alpha\)-bad for } S \text{ w.r.t \(A\) and \(f\)}}
    \le 4^{d} \cdot \left. \binom{\rho - \alpha t}{t - \alpha t} \middle/ \binom{\rho}{t} \right..
  \end{align}
\end{lemma}

\begin{proof}
  In the following, when we say that a pair \((s,s')\) is \(\alpha\)-bad for a subset \(S\),
  we mean that it is \(\alpha\)-bad for \(S\) \emph{with respect to \(f\) and \(A\)}.
  We will bound the probability that a fixed pair \((s,s')\) is \(\alpha\)-bad for \(S\),
  and then take a union bound over all pairs in \(\setSK^{2}\) to get \eqref{ineq:equiv}.
  So fix a pair \((s,s')\).
  We claim that \((s,s')\) can be \(\alpha\)-bad for at most \(\binom{\rho - \alpha t}{t - \alpha t}\) size \(t\) subsets of \(A\).
  To see this, first observe that the only size \(t\) subsets for which \((s, s')\) can be \(\alpha\)-bad
  are those containing every \(c \in A\) on which \(s\) and \(s'\) disagree.
  This follows from the definition of ``badness'':
  if \((s,s')\) is \(\alpha\)-bad for a subset \(S' \subset A\),
  then \(s\) and \(s'\) must agree on every \(c \in A\) which is outside of \(S'\),
  and hence they can only disagree \emph{inside} of the subset \(S'\).
  Now assume that there are exactly \(k\) distinct \(c \in A\) on which \(s\) and \(s'\)
  disagree, and denote the set of these \(c \in A\) by \(D\).
  Then, by the preceding argument, the number of size \(t\) subsets for which \((s, s')\)
  can be \(\alpha\)-bad equals the number of size \(t\) subsets of \(A\) containing the set \(D\).
  But this is exactly
  \[
    \binom{\abs{A} - \abs{D}}{t - \abs{D}} = \binom{\rho - k}{t - k}.
  \]
  Now, if \((s, s')\) is \(\alpha\)-bad for any size \(t\) subset at all, then \(k > \alpha t\). Hence
  \[
    \binom{\rho - k}{t - k} \le \binom{\rho - k}{t - \alpha t} \le \binom{\rho - \alpha t}{t - \alpha t},
  \]
  where the first inequality holds if we take
  \(
    t - \alpha t \le (\rho - k)/2
  \)
  (by the increasing property of binomial coefficients),
  for which it is sufficient to have
  \(
    t - \alpha t \le (\rho - t)/2,
  \)
  since \(t \ge k\).
  Taking e.g \(t \le (\rho - t)/2 \iff t \le \rho / 3\) ensures this.

  Now denote the family of all size \(t\) subsets \(S' \subset A\) for which \((s,s')\) is \(\alpha\)-bad by \(\mathcal{F}\).
  Then, as we have seen, \(\abs{F} \le \binom{\rho - \alpha t}{t - \alpha t}\).
  And since \(S\) is picked uniformly among the \(\binom{\rho}{t}\) size \(t\) subsets
  of \(A\), we have
  \begin{equation}
    \prob{(s,s') \text{ \(\alpha\)-bad for } S}
    = \sum_{S' \in \mathcal{F}}{\prob*{S = S'}}
    = \sum_{S' \in \mathcal{F}}{{\binom{\rho}{t}}^{-1}}
    \le {\binom{\rho}{t}}^{-1} \binom{\rho - \alpha t}{t - \alpha t}. \notag
  \end{equation}
  % where for the final inequality we have used that \(\abs{\mathcal{F}} \le \binom{\rho - \alpha t}{t - \alpha t}\).
  % the pair \((s, s')\) is
  % \(\alpha\)-bad for at most \(\binom{\rho - \alpha t}{t - \alpha t}\) size \(t\) subsets of \(A\).
  Taking a union bound over all pairs \((s,s') \in \setSK^{2}\) and using that \(\abs{\setSK^{2}} \le \abs{\{0,1\}^{2d}} = 4^{d}\),
  we get \eqref{ineq:equiv}.
\end{proof}

\begin{remark}
  \TODO{Say something about the lack of dependence on the function \(f\) in the lemma (it can be anything).}
\end{remark}

% In order for the bound in Lemma \ref{lemma:equiv} to be useful,
% we need the quotient of binomial coefficients in \eqref{ineq:equiv}
% to be significantly smaller than \(4^{d}\).
% This will depend on how we choose our parameters \(\rho, \alpha\) and \(t\).
% Importantly, we will be in a position to choose these parameters freely,
% so that the bound can indeed be made effective.
The following corollary bounds the quotient of binomial coefficients in \eqref{ineq:equiv}
trivially (\TODO{does it though?}), but will be good enough for our purposes.

\begin{corollary}\label{corollary:equiv}
  With the same setup as in Lemma \ref{lemma:equiv}, we have
  \begin{equation}\label{ineq:corollary:equiv}
    \prob*{\exists (s,s') \in \setSK^{2} : (s,s') \text{ is \(\alpha\)-bad for } S \text{ w.r.t \(A\) and \(f\)}}
    \le 4^{d} \cdot \left( \frac{t}{\rho - \alpha t + 1} \right)^{\alpha t}.
  \end{equation}
\end{corollary}

\begin{proof}
  This follows from the simple fact that for \(k \le n \le m\),
  \begin{align}
    \left. \binom{m - k}{n - k} \middle/ \binom{m}{n} \right.
    = \frac{(m-k)!}{(n-k)!(m-n)!} \frac{n!(m-n)!}{m!}
    = \frac{n!}{(n-k)!} \frac{(m-k)!}{m!} \notag
    & \le n^{k} \frac{(m-k)!}{m!} \notag\\
    & \le n^{k} (m-k+1)^{-k} \notag\\
    & = \left( \frac{n}{m - k + 1} \right)^{k}. \notag
  \end{align}
  Applying this inequality to \eqref{ineq:equiv} gives the result.
\end{proof}


\begin{theorem}[Bounding the security loss of an \OWECPA reduction]\label{thm:owecpa}
  Let \(\KEM = (\Setup, \KG, \Encaps, \Decaps)\) be a KEM
  with \((1-\delta)\)-correctness,
  ciphertext space \setC, encapsulation key space \keyspace
  and secret key space \(\setSK \subset \{0,1\}^{q(\lambda)}\) for some polynomial \(q\),
  where \(\lambda\) is the security parameter.
  Let \(\Simple = (\Init, \Respond, \Win,\epsilon(\cdot))\) be a simple computational assumption
  with corresponding game \(\SICA\) as in \TODO{ref.},
  and let \(\redR\) be a simple reduction
  taking adversaries against \OWECPA security in \KEM to adversaries against \SICA.
  Let \(\mu,\rho,t \in \N, \alpha > 0\) with \(t \le \rho/3\).
  Then there exists an (inefficient \TODO{?}) adversary \advA against \(\OWECPA\) security in \KEM
  with \(\mu\) users and \(\rho\) encapsulations per user,
  whose success probability is at least
  \begin{equation}\label{ineq:A}
    p_{\advA} \ge 1 - \alpha - \delta \mu \rho
    - 4^{q(\lambda)} \cdot \left( \frac{t}{\rho - \alpha t + 1} \right)^{\alpha t},
  \end{equation}
  and an adversary \redM in the game \(\SICA\) whose advantage \(\epsilon_{\redM}\) satisfies
  \begin{equation}\label{ineq:M}
    \abs{\epsilon_{\redR^{\advA}} - \epsilon_{\redM}} \le 1/\mu + \alpha
    + 4^{q(\lambda)} \cdot \left( \frac{t}{\rho - \alpha t + 1} \right)^{\alpha t},
  \end{equation}
  where \(\epsilon_{\redR^{\advA}}\) denotes the advantage of \(\text{ }\redR^{\advA}\) in \SICA.

  Moreover, the running time of \redM is polynomial in the parameters \((\mu, \rho)\),
  and if the running time of the reduction \(\redR^{A}\) is polynomial in \(\lambda\),
  where we consider the time taken to execute \(\advA\) to be constant,
  then the running time of the adversary \redM will be polynomial in \((\lambda,\mu,\rho)\).
\end{theorem}

\begin{corollary}[Impossibility of user-tight \OWECPA reductions]
  \TODO{write this.}
\end{corollary}

\begin{corollary}[Impossibility of encapsulation-tight \OWECPA reductions]
  \TODO{write this.}
\end{corollary}

\begin{corollary}[Impossibility of tight \INDECPA reductions]
  \TODO{write this.}
\end{corollary}

\begin{proof}[Proof of Theorem \ref{thm:owecpa}]
  We begin by describing the adversary \advA and the meta-reduction \redM.
  Let \(\rho, \mu, t \in \N, \alpha > 0\). Adversary \advA is given as follows.

  \begin{enumerate}[itemsep=0.1cm]
    \item Receive \(\params, \pk_{1}, \ldots, \pk_{\mu}\) from the \(\OWECPA(\KEM)\) challenger.
    \item Query \(\Enc(i)\) \(\rho\) times for each \(i \in [\mu]\) and receive encapsulations \((c_{i,j})_{i \in [\mu], j \in [\rho]}\).
    \item For each \(i \in [\mu]\), do the following:
          \begin{itemize}[label={\textbullet},itemsep=0.1cm]
            \item Pick a uniformly random size \(t\) subset \(S_{i} \subset [\rho]\).
            \item Query \(\Reveal(i,j)\) for each \(j \in [\rho] \setminus S_{i}\)
                  and receive keys \((k_{i,j})_{j \in [\rho] \setminus S_{i}}\).
          \end{itemize}
    \item\label{advA:corrupt} Sample \(i^{*} \sampleR [\mu]\). Then repeat the following for each \(i \in [\mu]\) with \(i \ne i^{*}\):
          \begin{itemize}[label={\textbullet},itemsep=0.1cm]
            \item Query \(\Corrupt(i)\) and receive \(\sk_{i}\).
            \item Abort if \(\KEM.\Decaps(\sk_{i},c_{i,j}) \ne k_{i,j}\) for some \(j \in [\rho] \setminus S_{i}\).
          \end{itemize}
     \item\label{advA:brute} Brute-force search for a candidate \(\sk_{i^{*}} \in \{0,1\}^{q(\lambda)}\) as follows:
          \begin{itemize}[label={\textbullet},itemsep=0.1cm]
            \item For each \(\sk \in \{0,1\}^{q(\lambda)}\),
                  check whether \(\KEM.\Decaps(\sk,c_{i^{*},j}) = k_{i^{*},j}\) for each \(j \in [\rho] \setminus S_{i^{*}}\).
            \item If so: set \(\sk_{i^{*}} := \sk\) and exit the loop. Otherwise: continue.
            \item If no \(sk_{i^{*}}\) was found during the brute-force search: abort.
          \end{itemize}
    \item Sample \(j^{*} \sampleR S_{i^{*}}\),
          and use \(\sk_{i^{*}}\) to compute a candidate key \(k_{i^{*}, j^{*}} \assign \KEM.\Decaps(\sk_{i^{*}},c_{i^{*}, j^{*}})\).
          Output \((i^{*}, j^{*}, k_{i^{*}, j^{*}})\) in the \OWECPA game.
  \end{enumerate}

  Now, for the meta-reduction \redM:

  \begin{enumerate}[itemsep=0.1cm]
    \item Receive the challenge \(c\) from the challenger in the game \SICA (recall Figure \ref{fig:simple}).
    \item Run \(\redR^{\oracle,\advA}(c)\) while simulating \advA towards \redR,
          until Step \ref{advA:corrupt} of \advA is reached.
          Here, \(\oracle\) denotes the oracle in \redM's instance of the game \SICA,
          which \redM uses to simulate the corresponding oracle in \redR's instance of the game.
    \item\label{advM:rewind} Starting from Step \ref{advA:corrupt} of \advA, run \(\redR^{\oracle}\) with every possible choice of \(i^{*} \in [\mu]\)
          (with the same randomness and state each time, except for the choice of \(i^{*}\)),
          and simulate \advA towards \(\redR^{\oracle}\)
          until the brute-force search in Step \ref{advA:brute} of \advA is reached.
          During the \(\ell\)-th rerun, \advA makes \(\Corrupt\)-queries to each user \(i \ne \ell\).
          For each such query, store the received secret key as \(\sk_{i}^{\ell}\).
    \item\label{advM:sample} Rerun \(\redR^{\oracle}\) for a final time, this time for a uniformly random \(i^{*} \sampleR [\mu]\),
          until Step \ref{advA:brute} of \advA is reached.
          If while simulating \advA in this step, \advA aborts,
          simply continue by running \redR to completion
          and output whatever \redR outputs in the game \SICA.
    \item\label{advM:key} Check if one of the secret keys \(\sk_{i^{*}}^{\ell}\) received during the rewinding
          from the previous steps satisfies that
          \(\KEM.\Decaps(sk_{i^{*}}^{\ell},c_{i^{*},j}) = k_{i^{*},j}\)
          for each \(j \in [\rho] \setminus S_{i^{*}}\).
          If so, pick one such secret key and denote it by \(\sk_{i^{*}}'\).
          If not: abort.
    \item Finally: use \(\sk_{i^{*}}'\) to simulate the final part of \advA's execution towards \(\redR^{\oracle}\).
          That is, pick \(j^{*} \sampleR S_{i^{*}}\)
          and compute \(k_{i^{*},j^{*}}' \assign \KEM.\Decaps(\sk_{i^{*}}',c_{i^{*},j^{*}})\).
          Then let \advA output \((i^{*},j^{*}, k_{i^{*},j^{*}}')\) to \redR.
          After this is done, let \(\redR^{\oracle}\) run until termination,
          and output whatever \(\redR^{\oracle}\) outputs in the game \SICA.
  \end{enumerate}

  Lemma \ref{lemma:A} will show that inequality \eqref{ineq:A} holds,
  and the running time of \redM is a simple calculation.
  It thus remains to show that
  \[
    \abs{\epsilon_{\redR^{\advA}} - \epsilon_{\redM}} \le 1/\mu + \alpha
    + 4^{q(\lambda)} \cdot \left. \binom{\rho - \alpha t}{t - \alpha t} \middle/ \binom{\rho}{t} \right.,
  \]
  where \(\epsilon_{\redR^{\advA}}\) and \(\epsilon_{\redM}\) denote
  the advantages of \(\redR^{\advA}\) and \redM, respectively, in the game \SICA
  against the simple computational assumption \(\Simple = (\Init, \Respond, \Win, \epsilon(\cdot))\).
  Let us denote the instances of the \SICA game played by \redM and \(\redR^{\advA}\)
  by \(\SICA_{\Simple}^{\redM}\) and \(\SICA_{\Simple}^{\redR}\), respectively.
  To simplify the analysis, we will define \(\SICA_{\Simple}^{\redM}\)
  and \(\SICA_{\Simple}^{\redR}\) over the same probability space.
  More precisely, we assume that the \(\Init\) algorithm of \Simple
  is given the same random coins at line \ref{code:simple:init} of Figure \ref{fig:simple}
  in either instance of the \SICA game.
  From this, it follows that the tuple \((\state_{\Simple},c)\) of state and challenge
  output by \(\Init\) is identical in \(\SICA_{\Simple}^{\redM}\) and \(\SICA_{\Simple}^{\redR}\).
  Now denote the challenge solutions output by \redM and \(\redR^{\advA}\)
  in \(\SICA_{\Simple}^{\redM}\) and \(\SICA_{\Simple}^{\redR}\)
  by \(s_{\redM}\) and \(s_{\redR}\), respectively (corresponding to line \ref{code:simple:output} of Figure \ref{fig:simple})
  (\TODO{refer to \(s\) as a challenge solution in the definition of \SICA}).
  We claim that
  \begin{equation}\label{ineq:owecpa:diff}
    \abs{\epsilon_{\redM} - \epsilon_{\redR^{\advA}}} \le \prob{s_{\redM} \ne s_{\redR}}.
  \end{equation}
  It follows that it will be enough to bound the probability \(\prob{s_{\redM} \ne s_{\redR}}\).
  We give such a bound in Lemma \ref{lemma:M}, from which the inequality \eqref{ineq:M} will follow.

  To see that \eqref{ineq:owecpa:diff} holds,
  observe that the winning condition in the game \SICA
  is defined by the deterministic algorithm \(\Win\)
  of the simple computational assumption \(\Simple\),
  which takes as input only the state \(\state_{\Simple}\) which was output by \(\Init\)
  and a challenge solution \(s\)
  (in particular, it does not depend on the queries made to the oracle \oracle).
  Since the states of \(\SICA_{\Simple}^{\redM}\) and \(\SICA_{\Simple}^{\redR}\)
  are equal by assumption, we find that if \(s_{\redM} = s_{\redR}\), then
  \[
  \SICA_{\Simple}^{\redR} \to 1 \iff \Win(\state_{\Simple},s_{\redM}) = 1 \iff \Win(\state_{\Simple},s_{\redM}) = 1 \iff \SICA_{\Simple}^{\redM} \to 1.
  \]
  (\TODO{change arrow notation?}).
  Hence, in this case, the two instances of the \SICA game will output the same bit \(b'\),
  and \eqref{ineq:owecpa:diff} follows by a standard ``identical-until-bad'' argument
  (\TODO{fill in the details.})
\end{proof}

\begin{lemma}\label{lemma:M}
  With the same setup as above,
  \begin{equation}
    \prob{s_{\redM} \ne s_{\redR}}
    \le 1/\mu + \alpha
    + 4^{q(\lambda)} \cdot \left( \frac{t}{\rho - \alpha t + 1} \right)^{\alpha t}.
  \end{equation}
\end{lemma}

\begin{proof}
  \TODO{State what we are actually doing here. Simulating the adversary towards the meta-reduction,
    and how this will ensure \(s_{\redR} = s_{\redM}\)}.

  We start by clarifying some terminology which we will use throughout the proof.
  Consider a user \(i \in [\mu]\). Let \(c_{i,1}, \ldots, c_{i,\rho}\) be the encapsulations
  on user \(i\) received by \advA from queries to \(\Enc(i)\) in the
  \(\OWECPA\) game and let \(S_{i} \subset [\rho]\) be a size \(t\) subset.
  Borrowing the terminology of Definition \ref{def:bad},
  we will call a pair of secret keys \((\sk, \sk') \in \bin^{q(\lambda)} \times \bin^{q(\lambda)}\) for the KEM \KEM
  \emph{\(\alpha\)-bad} for the subset \(S_{i}\) if \(\KEM.\Decaps(\sk, c_{i,j}) = \KEM.\Decaps(\sk', c_{i,j})\)
  for all \(j \in [\rho] \setminus S_{i}\),
  but \(\KEM.\Decaps(\sk, c_{i,j}) \ne \KEM.\Decaps(\sk', c_{i,j})\) for more than an \(\alpha\)-fraction of \(j \in S_{i}\).
  That is, the two secret keys \((\sk,\sk')\) are \(\alpha\)-bad if they agree
  on every ciphertext \(c_{i,j}\) with \(j\) outside of the set \(S_{i}\),
  yet they disagree on more than an \(\alpha\)-fraction of ciphertexts
  \(c_{i,j}\) with \(j\) inside of the set \(S_{i}\).

  From now on, let \eventE denote the event that \(s_{\redM} \ne s_{\redR}\).
  We partition \eventE into three events, which we define as follows:
  \begin{itemize}[label={\textbullet},itemsep=0.1cm]
    \item \eventi{1}: \redM aborts in Step \ref{advM:key},
          because none of the secret keys \(\sk_{i^{*}}^{\ell}\) received by \redM during Step \ref{advM:rewind}
          satisfy that \(\KEM.\Decaps(\sk_{i^{*}}^{\ell},c_{i^{*},j}) = k_{i^{*},j}\) for each \(j \in [\rho] \setminus S_{i^{*}}\).
    \item \eventi{2}: \redM receives a secret key \(\sk_{i^{*}}'\) such that
          \(\KEM.\Decaps(\sk_{i^{*}}',c_{i^{*},j}) = k_{i^{*},j}\) for each \(j \in [\rho] \setminus S_{i^{*}}\)
          (meaning that \eventi{1} does not occur),
          but the pair \((\sk_{i^{*}}, \sk_{i^{*}}')\) formed by \redM's secret key in \(\Simple'\)
          and the secret key found by \advA in its brute-force search (Step \ref{advA:brute} of \advA)
          in \Simple is \emph{\(\alpha\)-bad} for the subset \(S_{i^{*}}\).
    \item \eventi{3}: Neither \eventi{1} nor \eventi{2} occur, but \eventE still occurs.
  \end{itemize}

  By definition, \(\eventE \subset \eventi{1} \cup \eventi{2} \cup \eventi{3}\).
  We bound the probability of each event separately, beginning with \eventi{3}.

  If \eventi{3} occurs, then \eventi{1} does not, by definition.
  Thus, if \eventi{3} occurs, \redM must find a secret key \(\sk_{i^{*}}'\) such that
  \[
  \KEM.\Decaps(\sk_{i^{*}}',c_{i^{*},j}) = k_{i^{*},j} \text{ for each } j \in [\rho] \setminus S_{i^{*}}.
  \]
  Hence such a secret key exists, and so \advA is guaranteed to find such a secret key \(\sk_{i^{*}}\)
  in its brute-force search in \Simple as well (the two secret keys may not be the same).
  Also, the two games \Simple and \(\Simple'\) will proceed identically from the point of view of \redR,
  until it comes time for \advA to output a candidate key \(k_{i^{*},j^{*}}\)
  (note that this relies on the oracle \oracle being simulated correctly towards \(\redR^{\advA}\),
  which follows from the fact that the two games \(\Simple'\) and \Simple
  have identical state \(\state_{\Simple'} = \state_{\Simple}\),
  so that \(\Respond(\state_{\Simple'},m) = \Respond(\state_{\Simple},m)\) for every query \(m\)).
  \TODO{make this clearer.}
  Hence, the only way we can have \(s_{\redM} \ne s_{\redR}\)
  is if the key \(k_{i^{*},j^{*}}'\) computed by \redM differs from the key \(k_{i^{*},j^{*}}\) computed by \advA,
  i.e
  \[
    \KEM.\Decaps(\sk_{i^{*}}',c_{i^{*},j}) \ne \KEM.\Decaps(\sk_{i^{*}},c_{i^{*},j}).
  \]
  But since \eventi{2} also has not occurred (by the definition of \eventi{3}),
  this can happen with probability at most \(\alpha\).
  The reason is that, in this case, the pair \((\sk_{i^{*}},\sk_{i^{*}}')\) cannot be
  \(\alpha\)-bad for the subset \(S_{i^{*}}\)
  (by the definition of the event \eventi{2}).
  And we already know that
  \[
  \KEM.\Decaps(\sk_{i^{*}},c_{i^{*},j}) = k_{i^{*},j} = \KEM.\Decaps(\sk_{i^{*}}',c_{i^{*},j}) \text{ for each } j \in [\rho] \setminus S_{i^{*}}.
  \]
  Because \((\sk_{i^{*}}, \sk_{i^{*}}')\) is not \(\alpha\)-bad for \(S_{i^{*}}\),
  this means that we can only have
  \(\KEM.\Decaps(\sk_{i^{*}},c_{i^{*},j}) \ne \KEM.\Decaps(\sk_{i^{*}}',c_{i^{*},j})\)
  for at most an \(\alpha\)-fraction of \(j \in S_{i^{*}}\).
  Since \(j^{*}\) is picked uniformly at random from \(S_{i^{*}}\), the probability that
  \(\KEM.\Decaps(\sk_{i^{*}}',c_{i^{*},j}) \ne \KEM.\Decaps(\sk_{i^{*}},c_{i^{*},j})\)
  is thus at most \(\alpha\).
  In conclusion,
  \[
    \prob{\eventi{3}} \le \alpha.
  \]

  As for \eventi{1}, we have the bound
  \[
    \prob{\eventi{1}} \le 1/\mu.
  \]
  This follows from essentially the same type of argument as was used in the paper by \cite{EC:BJLS16} and several others.
  The intuition is that, if \redM fails to receive a secret key of the desired form,
  then with high probability over the choice of \(i^{*}\), \advA{} will abort,
  so that \redM's simulation of \advA succeeds trivially (and \eventi{1} does not occur).
  A more formal argument follows.

  Note that, since we are analyzing the execution of \redM and \(\redR^{\advA}\) over the same probability space,
  we will make the assumption that the choice of \(i^{*} \sampleR [\mu]\) in Step \ref{advM:sample} of \redM in \(\Simple'\)
  is the same as that in Step \ref{advA:corrupt} of the ``real'' execution of \advA by \(\redR^{\advA}\) in \Simple.

  If for this choice of \(i^{*}\), the simulation of \advA by \redM aborts (during Step \ref{advA:corrupt} of \advA),
  then \redM simply runs \redR to completion without doing any further simulation of \advA, and \eventi{1} cannot occur.
  Hence, for \eventi{1} to occur, \(i^{*}\) must be such that \advA does not abort, which by Step \ref{advA:corrupt} of \advA
  means that for every \(i \in [\mu] \setminus \{i^{*}\}\) we must have \(\KEM.\Decaps(\sk_{i}, c_{i,j}) = k_{i,j}\) for each \(j \in [\rho] \setminus S_{i}\).

  Now, if \eventi{1} occurs, then \redM fails to find a secret key satisfying this condition for user \(i^{*}\),
  during its rewinding of \advA.
  % But this means, that, for each rewind of \advA except the one in which the current \(i^{*}\) is picked,
  In particular, this means that, during the \(\ell\)-th rewind of \advA, for each \(\ell \ne i^{*}\),
  the secret key \(\sk_{i^{*}}^{\ell}\) received by \advA from its query to \(\Corrupt(i^{*})\)
  does not satisfy this condition, causing \advA to abort in this particular rewind.
  Thus, the only way that \eventi{1} can occur is if
  for every choice \(i^{*} \in [\mu]\) except the current one, \advA will abort.
  Since \(i^{*}\) was chosen uniformly at random from \([\mu]\),
  the probability of this happening is at most \(1/\mu\).

  Finally, we bound the probability of the event \eventi{2}.
  For this, we will apply Lemma \ref{lemma:equiv}.
  Recall that \eventi{2} is the event that
  \redM receives a secret key \(\sk_{i^{*}}'\) such that
  \[
  \KEM.\Decaps(\sk_{i^{*}}',c_{i^{*},j}) = k_{i^{*},j} \text{ for each } j \in [\rho] \setminus S_{i^{*}},
  \]
  but the pair \((\sk_{i^{*}}, \sk_{i^{*}}')\) formed by \redM's secret key and the secret key found by \advA
  is \emph{\(\alpha\)-bad} for the subset \(S_{i^{*}}\).
  As before, since \redM finds a secret key \(\sk_{i^{*}}'\) in \(\Simple'\) satisfying that
  \(\KEM.\Decaps(\sk_{i^{*}}',c_{i^{*},j}) = k_{i^{*},j} \text{ for each } j \in [\rho] \setminus S_{i^{*}}\),
  \advA will find such a secret key in \Simple as well, so it makes sense to speak of the pair \((\sk_{i^{*}}, \sk_{i^{*}}')\).
  From the definition of \eventi{2}, it should be clear that
  \begin{equation}\label{ineq:M:bad}
    \prob*{\eventi{2}} \le \prob{\text{There exists a pair } (\sk,\sk') \text{ which is \(\alpha\)-bad for } S_{i^{*}}}.
  \end{equation}
  Now we can apply Lemma \ref{lemma:equiv} to bound the right hand side of \eqref{ineq:M:bad}.
  To do this, we will need to assign the sets \(\setSK, \setC, \keyspace\) referred to in the lemma,
  and also pick the function \(f : \setSK \times \setC \to \keyspace\),
  the set \(A \subset \setC\) of ciphertexts and the parameters \(d,\rho,t,\alpha\).
  We will let the sets \(\setSK, \setC, \keyspace\)
  be the same as they are here (i.e they will be the secret key space, ciphertext space and encapsulation key space of the KEM \KEM), and likewise for the parameters \(\rho, t, \alpha\).
  We then set \(d = q(\lambda)\),
  and let the function \(f\) be the decapsulation function \(\KEM.\Decaps\).
  By our definition of a KEM (\TODO{define}), the decapsulation function
  \(\KEM.\Decaps\) is a deterministic function from \(\setSK \times \setC\) to \(\keyspace\),
  and hence it makes sense to use it in our application of Lemma \ref{lemma:equiv}.
  Now we let \(A\) be the set of ciphertexts received on user \(i^{*}\), that is
  \[
    A = \{c_{i^{*},1},c_{i^{*},2},\ldots,c_{i^{*},\rho}\},
  \]
  and we set \(S\) to be the set of ciphertexts in \(A\) indexed by \(S_{i^{*}}\).
  With this, Lemma \ref{lemma:equiv} along with Corollary \ref{corollary:equiv} give us
  \[
    \prob{\exists (\sk,\sk') \text{ which is \(\alpha\)-bad for } S_{i^{*}}}
    \le 4^{d} \cdot \left( \frac{t}{\rho - \alpha t + 1} \right)^{\alpha t}.
  \]

  Note that, for the application of Lemma \ref{lemma:equiv} to be formally correct here,
  we would actually need to verify that there are no duplicate ciphertexts
  \(c_{i^{*},j} = c_{i^{*},j'}\) among the ciphertexts received on user \(i^{*}\).
  Otherwise, the set \(A\) may not have size \(\rho\)
  (it could then be strictly smaller than \(\rho\)),
  and the set \(S\) could likewise be smaller than \(t\),
  so that the lemma would not apply.
  Now, for any secure KEM, we expect the probability of such a
  duplicate \(c_{i^{*},j} = c_{i^{*},j'}\) to be negligible in the real \OWECPA game.
  Nevertheless, we cannot assume that the same will hold
  when the reduction \redR acts as the challenger,
  as \redR could conceivably produce such duplicates regardless
  (recall that the reduction is free to do whatever it wishes, and does not need to
  faithfully emulate the real challenger in the \OWECPA game).
  However, if \advA ever receives such a duplicate \(c_{i^{*},j} = c_{i^{*},j'}\),
  then it can trivially win the \OWECPA game by revealing one of the ciphertexts and attacking the other.
  In this case, the adversary \advA's attack becomes efficient,
  and so we can simulate it efficiently towards \redR without issue.
  Hence we can ignore this technical detail.

  Combining the bounds for \(\prob{\eventi{1}}\), \(\prob{\eventi{2}}\) and \(\prob{\eventi{3}}\), we now get
  \begin{align}\label{ineq:E}
    \prob{\eventE} \le 1 / \mu + \alpha
    + 4^{d} \cdot \left( \frac{t}{\rho - \alpha t + 1} \right)^{\alpha t}. && (t \le \rho / 3),
  \end{align}
  which proves the lemma.
\end{proof}

\begin{lemma}[Lower bounding the success probability of \advA]\label{lemma:A}
  With the same setup as above, we have
  \begin{equation}
    p_{\advA} \ge 1 - \alpha - \delta \mu \rho
    - 4^{d} \cdot \left( \frac{t}{\rho - \alpha t + 1} \right)^{\alpha t},
  \end{equation}
  where \(p_{\advA}\) denotes the probability that \TODO{write.}
\end{lemma}

\begin{proof}
  Let us first bound the probability that \advA aborts before outputting anything.
  This can happen in two ways:
  \begin{enumerate}
    \item\label{advA:abort1} One of the secret keys \(\sk_{i}\) received by \advA
          via a \(\Corrupt(i)\) query does not satisfy that \(\KEM.\Decaps(\sk_{i}, c_{i,j}) = k_{i,j}\)
          for every \(j \in [\rho] \setminus S_{i}\).
    \item\label{advA:abort2} During \advA's brute-force search, \advA fails to find a candidate secret key \(\sk_{i^{*}}\)
          such that \(\KEM.\Decaps(\sk_{i^{*}}, c_{i^{*},j}) = k_{i^{*},j}\) for every \(j \in [\rho] \setminus S_{i^{*}}\).
  \end{enumerate}
  Since we are in the real game with an honest challenger,
  all of the secret keys \(\sk_{i}\) used by the challenger
  were validly generated, and the same holds for the encapsulations \(c_{i,j}\).
  Thus, by \((1 - \delta)\)-correctness,
  we will have \(\KEM.\Decaps(\sk_{i},c_{i,j}) = k_{i,j}\) for every \(i \in [\mu]\) and \(j \in \rho\)
  except with probability \(\delta \mu \rho\),
  where \(k_{i,j}\) denotes the key that was generated along with \(c_{i,j}\) by the challenger.
  In this case, clearly \ref{advA:abort1} cannot occur.
  Likewise, \ref{advA:abort2} will not occur,
  since the secret key \(\sk_{i^{*}}\) held by the challenger for user \(i^{*}\)
  satisfies that \(\KEM.\Decaps(\sk_{i^{*}}, c_{i,j}) = k_{i,j} \text{ for every } j \in [\rho] \setminus S_{i^{*}}\).
  Because of this, a secret key \(\sk_{i^{*}}\) of the desired form exists,
  and \advA's brute-force search will not fail.
  Thus, the probability that \advA aborts before outputting anything is at most \(\delta \mu \rho\).

  Now, if neither \ref{advA:abort1} nor \ref{advA:abort2} occur,
  then \advA wins so long as the key
  \[
  k_{i^{*},j^{*}}' \assign \KEM.\Decaps(\sk_{i^{*},j^{*}}, c_{i^{*},j^{*}})
  \]
  computed by \advA equals the true key corresponding to the encapsulation \(c_{i^{*},j^{*}}\) in the game.
  Applying Lemma \ref{lemma:equiv},
  we find that the pair of secret keys for user \(i^{*}\) held by \advA
  and the challenger in the \(\OWECPA\) game is
  \(\alpha\)-bad for the subset \(S_{i^{*}}\) with probability at most
  \(
    4^{q(\lambda)} \cdot \left( \frac{t}{\rho - \alpha t + 1} \right)^{\alpha t}.
  \)
  Thus the two secret keys will produce identical decapsulations of \(c_{i^{*},j^{*}}\), except with probability
  \(
    \alpha +
    4^{q(\lambda)} \cdot \left( \frac{t}{\rho - \alpha t + 1} \right)^{\alpha t}.
  \)
  Also, we may assume as before that
  \(\KEM.\Decaps(\sk_{i},c_{i,j}) = k_{i,j}\) for every \(i \in [\mu]\) and \(j \in \rho\),
  from which we find that the decapsulation of \(c_{i^{*},j^{*}}\)
  under the challenger's secret key for user \(i^{*}\)
  equals the true key in the \(\OWECPA\) game.
  Hence in this case, \advA wins except with probability
  \(
  \alpha
  + 4^{q(\lambda)} \cdot \left( \frac{t}{\rho - \alpha t + 1} \right)^{\alpha t}.
  \)

  Summing the failure probabilities, we now get
  \[
    p_{\advA} \ge 1 - \alpha - \delta \mu \rho
    - 4^{q(\lambda)} \cdot \left( \frac{t}{\rho - \alpha t + 1} \right)^{\alpha t},
  \]
  as desired.
\end{proof}


%%% Local Variables:
%%% mode: latex
%%% TeX-master: "../main"
%%% End:



\section{Further applications of the meta-reduction}

\subsection{An impossibility result for authenticated key exchange}

\begin{theorem}[Bounding the security loss of an \OwFSst reduction]\label{thm:owfsst}
  Let \(\prot{} = (\KG, \Init, \Run)\) be a key exchange protocol
  with \((1-\delta)\)-correctness,
  message space \setM, session state space \setST, session key space \keyspace
  and secret key space \(\setSK \subset \{0,1\}^{q(\lambda)}\) for some polynomial \(q\),
  where \(\lambda\) is the security parameter.
  Let \(\Simple = (\Init, \Respond, \Win)\) be the game associated to a simple computational assumption,
  and let \(\redR : \OwFSst(\prot) \to \Simple\) be a simple reduction
  taking \(\OwFSst(\prot)\) adversaries to adversaries against \Simple.
  Let \(\mu,\rho,t \in \N, \alpha > 0\) with \(t \le \rho/3\).
  Then there exists an (inefficient) adversary \advA in the \(\OwFSst(\prot)\) game
  with \(\mu\) users and \(2\rho\) sessions per user, whose success probability is at least
  \begin{equation}\label{ineq:owfsst:A}
    p_{\advA} \ge 1 - \alpha - \delta \mu \rho
    - 4^{d} \cdot \left( \frac{t}{\rho - \alpha t + 1} \right)^{\alpha t},
  \end{equation}
  and an adversary \redM in the game \(\Simple\)
  whose advantage \(\epsilon_{\redM}\) satisfies
  \begin{equation}\label{ineq:owfsst:M}
    \abs{\epsilon_{\redR^{\advA}} - \epsilon_{\redM}} \le 1/\mu + \alpha
    + 4^{d} \cdot \left( \frac{t}{\rho - \alpha t + 1} \right)^{\alpha t},
  \end{equation}
  where \(\epsilon_{\redR^{\advA}}\) denotes the advantage of \(\text{ }\redR^{\advA}\) in \Simple.

  Moreover, the running time of \redM is polynomial in the parameters \((\mu, \rho)\),
  and if the running time of the reduction \(\redR^{A}\) is polynomial in \(\lambda\),
  where we consider the time taken to execute \(\advA\) to be constant,
  then the running time of the adversary \redM will be polynomial in \((\lambda,\mu,\rho)\).
\end{theorem}

\begin{proof}
  Let \(\ell\) be a bound on the number of times an initiator session
  is invoked (via a call to \Init or \Run) in an honest execution of the protocol \prot{}.
  Let \(f : \setSK \times \setM^{l} \times \setST^{l} \to \keyspace \)
  denote the function that, given as input a secret key \(\sk\),
  \(\ell\) responder messages \((m^{(k)})_{k \in [\ell]} \subset \setM\) (some of which may be empty)
  and \(\ell\) initiator session states \((\state^{(k)})_{k \in [\ell]}\),
  outputs the session key that an initiator session would produce
  (given the above secret key, states and response messages).
  By our definition of a key exchange protocol (\TODO{define}),
  this function is well-defined, since
  any session key can be deterministicallly produced given the secret key, states and messages of a session.
  Moreover, the function can be computed efficiently by invoking the \Init and \Run procedures of \prot{}.

  We begin by describing the adversary \advA. It is given as follows:
  \begin{enumerate}[itemsep=0.1cm]
    \item Receive \(\params, \pk_{1}, \ldots, \pk_{\mu}\) from the \(\OwFSst(\prot)\) challenger.
    \item For each \(i \in [\mu]\), do the following:
      \begin{itemize}[label={\textbullet},itemsep=0.1cm]
        \item Let \(i' = i + 1 \mod{\mu}.\)
        \item For each \(j \in [\rho]\), start an initiator session \(\sID_{i,j}^{I}\) at user \(i\)
              and a responder session \(\sID_{i,j}^{R}\) at user \(i'\)
              and forward messages between the two sessions until both of them accept
              (aborting in the case that either session does not accept).
        \item For each of the at most \(\ell\) times the initiator session \(\sID_{i,j}^{I}\) is invoked,
              reveal its current state through a call to \(\RevState(\sID_{i,j}^{I})\).
              Store the resulting tuple of session states of \(\sID_{i,j}^{I}\) as \((\state_{i,j}^{(k)})_{k \in [\ell]}\).
        \item Also, store the messages received by \(\sID_{i,j}^{I}\) as \((m_{i,j}^{(k)})_{k \in [\ell]}\).
      \end{itemize}
    \item For each \(i \in [\mu]\), repeat the following:
          \begin{itemize}[label={\textbullet},itemsep=0.1cm]
            \item Pick a uniformly random size \(t\) subset \(S_{i} \subset [\rho]\).
            \item Query \(\Reveal(\sID_{i,j}^{I})\) on each initiator session \(\sID_{i,j}^{I}\)
                  with \(j \in [\rho] \setminus S_{i}\),
                  and receive session keys \((K_{i,j})_{j \in [\rho] \setminus S_{i}}\).
          \end{itemize}
    \item\label{owfsst:advA:corrupt} Sample \(i^{*} \sampleR [\mu]\). Then repeat the following for each \(i \in [\mu]\) with \(i \ne i^{*}\):
          \begin{itemize}[label={\textbullet},itemsep=0.1cm]
            \item Query \(\Corrupt(i)\) and receive \(\sk_{i}\).
            \item For each \(j \in [\rho] \setminus S_{i}\),
                  compute \(K_{i,j}' \assign f(\sk_{i},(m_{i,j}^{(k)})_{k \in [\ell]},(\state_{i,j}^{(k)})_{k \in [\ell]})\)
                  and check whether \(K_{i,j}' = K_{i,j}\).
            \item If any of the above checks fail: abort.
          \end{itemize}
     \item\label{owfsst:advA:brute} Brute-force search for a candidate \(\sk_{i^{*}} \in \{0,1\}^{q(\lambda)}\) as follows:
          \begin{itemize}[label={\textbullet},itemsep=0.1cm]
            \item For each \(\sk \in \{0,1\}^{q(\lambda)}\),
                  check whether \(K_{i^{*},j}' = K_{i^{*},j}\) for each \(j \in [\rho] \setminus S_{i^{*}}\),
                  where \(K_{i^{*},j}' \assign f(\sk,(m_{i^{*},j}^{(k)})_{k \in [\ell]},(\state_{i^{*},j}^{(k)})_{k \in [\ell]})\).
            \item If so: set \(\sk_{i^{*}} := \sk\) and exit the loop. Otherwise: continue.
            \item If no \(sk_{i^{*}}\) was found during the brute-force search: abort.
          \end{itemize}
    \item Sample \(j^{*} \sampleR S_{i^{*}}\),
          and use \(\sk_{i^{*}}\) to compute a candidate key
          \(K_{i^{*}, j^{*}} \assign f(\sk_{i^{*}},(m_{i^{*},j^{*}}^{(k)})_{k \in [\ell]},(\state_{i^{*},j^{*}}^{(k)})_{k \in [\ell]})\).
          Output \((\sID_{i^{*},j^{*}}^{I}, K_{i^{*}, j^{*}})\) in the \OwFSst game.
  \end{enumerate}

  The meta-reduction \redM is almost identical to the meta-reduction of Theorem \ref{thm:owecpa}.
  We give it here for the sake of completeness.
  \begin{enumerate}[itemsep=0.1cm]
    \item Receive the challenge \(c\) from the challenger in the game \Simple.
    \item Run \(\redR^{\oracle,\advA}(c)\) while simulating \advA towards \redR,
          until Step \ref{owfsst:advA:corrupt} of \advA is reached.
    \item\label{owfsst:advM:rewind} Starting from Step \ref{owfsst:advA:corrupt} of \advA,
          run \(\redR^{\oracle}\) with every possible choice of \(i^{*} \in [\mu]\)
          and simulate \advA towards \(\redR^{\oracle}\)
          until the brute-force search in Step \ref{owfsst:advA:brute} of \advA is reached.
          During the \(k\)-th rerun for each \(k \in [\mu]\), make sure to store the output of
          the \(\Corrupt\)-queries made by \advA to each user \(i \ne k\) as \(\sk_{i}^{k}\).
    \item\label{owfsst:advM:sample} Rerun \(\redR^{\oracle}\) for a final time,
          this time for a uniformly random \(i^{*} \sampleR [\mu]\),
          until Step \ref{owfsst:advA:brute} of \advA is reached.
          If while simulating \advA in this step, \advA aborts,
          simply continue by running \redR to completion
          and output whatever \redR outputs in the game \Simple.
    \item\label{owfsst:advM:key} Check if one of the secret keys \(\sk_{i^{*}}^{k}\)
          received during the rewinding from the previous steps satisfies that
          \(K_{i^{*},j}' = K_{i^{*},j}\) for each \(j \in [\rho] \setminus S_{i^{*}}\),
          where \(K_{i^{*},j}' \assign f(\sk_{i^{*}},(m_{i^{*},j}^{(k)})_{k \in [\ell]},(\state_{i^{*},j}^{(k)})_{k \in [\ell]})\).
          If so, pick one such secret key and denote it by \(\sk_{i^{*}}'\).
          If not: abort.
    \item Finally: use \(\sk_{i^{*}}'\) to simulate the last part of \advA's execution towards \(\redR^{\oracle}\).
          That is, pick \(j^{*} \sampleR S_{i^{*}}\) and use \(\sk_{i^{*}}\) to compute
          \(K_{i^{*}, j^{*}} \assign f(\sk_{i^{*}},(m_{i^{*},j^{*}}^{(k)})_{k \in [\ell]},(\state_{i^{*},j^{*}}^{(k)})_{k \in [\ell]})\).
          Then let \advA output \((\sID_{i^{*},j^{*}}^{I}, K_{i^{*}, j^{*}})\) to \redR.
          Afterwards, continue running \(\redR^{\oracle}\) until termination,
          and output whatever \(\redR^{\oracle}\) outputs in the game \Simple.
  \end{enumerate}

  Since the proof is very similar to the proof of Theorem \ref{thm:owecpa},
  we will not redo the details in full here,
  but rather emphasize the differences to the proof of Theorem \ref{thm:owecpa}
  and in particular how we apply Lemma \ref{lemma:equiv}.
  We begin by addressing the claim on the advantage of \redM in \eqref{ineq:owfsst:M}.
  Similarly to the proof of Theorem \ref{thm:owecpa},
  this reduces to bounding the probability that the simulation of \advA by \(\redM\)
  and the real execution of \advA by \(\redR^{A}\) will output different session keys
  \(K_{i^{*},j^{*}}' \ne K_{i^{*},j^{*}}\).
  As before, we see that, except with probability \(1/\mu\),
  \redM will obtain a secret key \(\sk_{i^{*}}'\) in Step \ref{owfsst:advM:key}
  which ``agrees'' with the revealed session keys \(K_{i^{*},j}\) for \(j \in [\rho] \setminus S_{i^{*}}\)
  (or \redM will end up simulating \advA trivially because \advA aborts).
  More precisely, in this case, \redM will obtain a secret key \(\sk_{i^{*}}'\) such that
  \[
    K_{i^{*},j} = f(\sk_{i^{*}}',(m_{i^{*},j}^{(k)})_{k \in [\ell]},(\state_{i^{*},j}^{(k)})_{k \in [\ell]})
    \text{ for each } j \in [\rho] \setminus S_{i^{*}}.
  \]
  Hence such a secret key exists, and \advA finds one in its brute-force search as well.
  Denoting the secret key of \advA by \(\sk_{i^{*}}\), we get
  \[
    f(\sk_{i^{*}},(m_{i^{*},j}^{(k)})_{k \in [\ell]},(\state_{i^{*},j}^{(k)})_{k \in [\ell]})
    = f(\sk_{i^{*}}',(m_{i^{*},j}^{(k)})_{k \in [\ell]},(\state_{i^{*},j}^{(k)})_{k \in [\ell]})
    \text{ for each } j \in [\rho] \setminus S_{i^{*}}.
  \]
  In other words, using the terminology of Definition \ref{def:agree},
  the two secret keys \((\sk_{i^{*}},\sk_{i^{*}}')\) of the adversary \advA and meta-reduction \redM
  agree with respect to \(f\) on every tuple
  \(((m_{i^{*},j}^{(k)})_{k \in [\ell]},(\state_{i^{*},j}^{(k)})_{k \in [\ell]})\) with \(j \in [\rho] \setminus S_{i^{*}}\).
  Now we can apply Lemma \ref{lemma:equiv}.
  For this, we will take the set \(\setC\) referred to in Lemma \ref{lemma:equiv}
  to be \(\setM^{\ell} \times \setST^{l}\)
  and define the set \(A \subset \setC\) as the set of message-state tuples of the above form on user \(i^{*}\).
  More precisely, we will let
  \[
    A =
    \{
    ((m_{i^{*},j}^{(k)})_{k \in [\ell]},(\state_{i^{*},j}^{(k)})_{k \in [\ell]}) : j \in [\rho]
    \}.
  \]
  Also, we take \(S\) to be the subset of \(A\) indexed by the set \(S_{i^{*}}\).
  Applying the lemma along with Corollary \ref{corollary:equiv} now gives us the bound
  \[
    \prob*{(\sk_{i^{*}},\sk_{i^{*}}') \text{ \(\alpha\)-bad for } S_{i^{*}}}
    \le 4^{d} \cdot \left( \frac{t}{\rho - \alpha t + 1} \right)^{\alpha t}.
  \]
  Finally, we find that, if \((\sk_{i^{*}},\sk_{i^{*}}')\) is not \(\alpha\)-bad for \(S_{i^{*}}\),
  then the simulation succeeds except with probability \(\alpha\).
  The bound \eqref{ineq:owfsst:M} now follows.

  As for the bound on the success probability of \advA in \eqref{ineq:owfsst:A},
  we can argue as follows.
  Notice that \advA may abort if any of the \(\mu \rho\) sessions fail to accept:
  by \((1-\delta)\)-correctness, this happens only with probability \(\delta \mu \rho\).
  This is in fact the only way that \advA will abort,
  since the checks done in Step \ref{owfsst:advA:corrupt}
  and the brute-force search in Step \ref{owfsst:advA:corrupt}
  are both guaranteed to succeed,
  unlike the case in the proof of Theorem \ref{thm:owecpa},
  where these steps could fail due to correctness errors.
  The reason is that, (\TODO{explain this}).

  Given this, it will be enough to argue that the secret key \(\sk_{i^{*}}'\) for user \(i^{*}\)
  found by \advA's brute-force search and
  the real secret key \(\sk_{i^{*}}\) of user \(i^{*}\) held by the challenger
  are likely to produce the same session key \(K_{i^{*},j^{*}}\),
  given the state and messages \(((m_{i^{*},j^{*}}^{(k)})_{k \in [\ell]},(\state_{i^{*},j^{*}}^{(k)})_{k \in [\ell]})\).
  Whenever this is the case, \advA wins the \OwFSst game.
  Indeed, the session \(\sID_{i^{*},j^{*}}^{I}\) picked by the adversary is valid in the \OwFSst game,
  since it has a single matching session \(\sID_{i^{*},j^{*}}^{R}\) which \TODO{blabla.}
  To bound the probability that
  \(\sk_{i^{*}}\) and \((\sk_{i^{*}}')\) produce different session keys \(K_{i^{*},j^{*}}' \ne K_{i^{*},j^{*}}\),
  we again apply Lemma \ref{lemma:equiv},
  which explains the \(\alpha\) and
  \(
    4^{d} \cdot \left( \frac{t}{\rho - \alpha t + 1} \right)^{\alpha t}
  \)
  terms in \eqref{ineq:owfsst:A}.
\end{proof}



%%% Local Variables:
%%% mode: latex
%%% TeX-master: "../main"
%%% End:


\appendix


\section{Supporting material}


\begin{namedproof}{Proving that \(\ell = \Omega(\rho^{1-\epsilon})\)}
  Let \(\epsilon \in (0,1)\) be given
  and let \(d \in \R\), to be determined later.
  Set \(\mu = \lambda^{d}\), \(\rho = 6 \lambda^{d} q(\lambda)\), \(t = \lambda^{d} q(\lambda)\),
  and \(\alpha = 1/\mu = \lambda^{-d}\).
  Then \(T_{\redM} \in \poly\), and also
  \begin{align}
    4^{q(\lambda)} \cdot \left( \frac{t}{\rho - \alpha t} \right)^{\alpha t}
    =
    4^{q(\lambda)} \cdot \left( \frac{\lambda^{d} q(\lambda)}{6\lambda^{d} q(\lambda) - q(\lambda)} \right)^{q(\lambda)}
    & = 4^{q(\lambda)} \cdot \left( \frac{1}{6 - \lambda^{-d}} \right)^{q(\lambda)} \notag\\
    & = \mathcal{O}((4 / 5)^{q(\lambda)})\notag\\
    & = \negl \notag.
  \end{align}
  We then find that
  \(\epsilon_{\advA} = \Omega(1)\), and
  \begin{align}\label{ineq:app}
    \epsilon_{\redR^{\advA}} \le \epsilon_{\redM} + 1/\mu + \alpha + \negl
    = 2/\mu + \negl
    = 2\lambda^{-d} + \negl.
  \end{align}
  Now let \(k \in \R_{\ge 0}\) be such that \(q(\lambda) = \mathcal{O}(\lambda^{k})\).
  Set \(d\) large enough that \((d+k)(1-\epsilon) \le d\)
  (it is enough to take \(d \ge k/\epsilon\) for this) (\TODO{redo the math}).
  Then we get
  \[
    \lambda^{d} \ge (\lambda^{d+k})^{1-\epsilon} = (\lambda^{d} \Omega(q(\lambda)))^{1-\epsilon}
  \]
  where we used that \(\lambda^{k} = \Omega(q(\lambda))\).
  Recalling that \(\rho = \lambda^{d} q(\lambda)\), we see that
  \[
    \lambda^{d} \ge (\Omega(\rho))^{1-\epsilon} = \Omega(\rho^{1-\epsilon}).
  \]
  Inserting this into \eqref{ineq:app}, we get
  \[
    \epsilon_{\redR^{\advA}} \le 2\lambda^{-d} + \negl
    = \mathcal{O}(\rho^{\epsilon-1}) + \negl
    = \mathcal{O}(\rho^{\epsilon-1}),
  \]
  and hence
  \[
    \epsilon_{\advA} / \epsilon_{\redR^{\advA}} = \Omega(\rho^{1-\epsilon}).
  \]
  Now \(\ell = \Omega(\rho^{1-\epsilon})\) follows by the fact that \(T_{\redR^{\advA}} \ge T_{\advA}\).
\end{namedproof}



%%% Local Variables:
%%% mode: latex
%%% TeX-master: "../main"
%%% End:


\bibliographystyle{splncs04} % ??????
\bibliography{cryptobib/abbrev3,cryptobib/crypto,refs}

\end{document}
