
\section{Preliminaries}

\subsection{Key exchange mechanisms}

\TODO{Add definition of a KEM.}

\begin{definition}[\((1-\delta)\)-correctness of a KEM]\label{def:corr}
  Let \(\KEM = (\Setup, \KG, \Encaps, \Decaps)\) be a KEM
  and let \(\delta : \mathbb{N} \to \mathbb{R}\) be a function of the security parameter.
  Then \KEM is said to have \emph{\((1-\delta)\)-correctness} if
  \begin{equation}
    \prob{\Decaps{}(\sk,c) \ne k : \params \assign \Setup{}(1^{\lambda}), (\sk,\pk) \assign \KG{}(\params), (c,k) \assign \Encaps{}(\pk)} \le \delta (\lambda)
  \end{equation}
\end{definition}


\begin{definition}[OW-ECPA security]\label{def:OWECPA_sec}
  Let \(\KEM = (\Setup, \KG, \Encaps, \Decaps)\) be a KEM
  and let \advA{} be an adversary in the \OWECPA{} game against \KEM (see Figure \ref{fig:ecpa}).
  Then the advantage of \advA{} against \OWECPA in \KEM is given by
  \begin{equation}
    \abs*{\prob{\OWECPA{}_{\KEM}^{\advA}(\mu,\rho,\lambda) \implies 1} - 1/\abs{\keyspace}},
  \end{equation}
  where \(\keyspace = \keyspace{}(\params)\) is the encapsulation key space of \KEM.
  \TODO{advantage notation?}
\end{definition}

\begin{figure}
  \begin{pcvstack}[center,space=0.5cm]
    \proc{Game $\OWECPA_{\KEM}^{\advA}(\mu, \rho, \lambda)$}{
      L_{rev}, L_{corr} \assign \List.\\
      \params \assign \KEM{}.\Setup{}(1^{\lambda}).\\
      \pcfor i \in [\mu]:\\
      \t (\sk_i,\pk_i) \assign \KG(\params).\\
      \oracle := \{\Enc, \Reveal, \Corrupt\}.\\
      (i^*, j^*, k^*) \assign \advA^{\oracle}(\params,\pk_1, \ldots, \pk_{\mu}).\\
      \pcif i^{*} \notin L_{corr} \land i^{*} \notin L_{rev} \land k^{*} = k_{i^{*},j^{*}}:\\
      \t \pcreturn 1.\\
      \pcelse:\\
      \t \pcreturn 0.
    }
    \proc{Oracle \Enc{}$(i)$}{
      \text{Assume this is the \(j\)-th query to \Enc{} on user \(i\) with } j \in [\rho].\\
      (c_{i,j}, k_{i,j}) \assign \Encaps{}(\pk_i).\\
      \pcreturn c_{i,j}.
    }
    \begin{pchstack}[space=1cm]
      \proc{Oracle \Reveal{}$(i,j)$}{
        L_{rev} \assign L_{rev} \cup \{(i,j)\}.\\
        \pcreturn k_{i,j}.
      }
      \proc{Oracle \Corrupt{}$(i)$}{
        L_{corr} \assign L_{corr} \cup \{i\}.\\
        \pcreturn \sk_i.
      }
    \end{pchstack}
  \end{pcvstack}\caption{}\label{fig:ecpa}
\end{figure}

\TODO{define simple reductions.}

\subsection{Computational assumptions}

When we prove our tightness impossibility results,
we will only be able to rule out tight reductions from a specific class of computational assumptions.
That is, we can only show that a tight reduction to a certain security game \gameG is impossible
if the assumption being reduced from is sufficiently restricted.
This is in some sense an inherent limitation, as we could not rule out a tight reduction to the game \gameG
from \emph{any} computational assumption
(a standard counterexample to this is to use the assumption that the game \gameG is secure.
Clearly there is a tight reduction from this assumption to the game \gameG).
Hence we need the assumption being reduced from to be ``weaker''
than the game \gameG, in a certain sense.

In the impossibility results of \TODO{cite}, the class of computational assumptions
in question are the so-called \emph{non-interactive} assumptions.
These are modeled by experiments in which an adversary receives a challenge \(c\)
and must output a solution \(s\) to the challenge, with no further interaction between the adversary and the experiment.
Following an observation of \TODO{cite}, we will actually slightly expand the class of computational assumptions
from which we can rule out tight reductions, at essentially no cost to the complexity of our proofs.

In \TODO{cite}, it is shown that a tight, simple reduction from the \emph{Strong Diffie-Hellman} assumption
to the weak forward secrecy of a certain class of key exchange protocols cannot exist,
unless the Strong Diffie-Hellman assumption is itself false.
The key point here is that the Strong Diffie-Hellman game is interactive,
and hence does not fit into the class of non-interactive assumptions considered in e.g \TODO{cite}.
Nevertheless \TODO{cite} are able to apply a similar meta-reduction strategy
to the one pioneered in \TODO{cite} to prove their impossibility result.
We observe that the Strong Diffie-Hellman game satisfies the following two criteria,
which allow the proof of \TODO{cite} to go through.
The first is that the interaction in the Strong Diffie-Hellman game is in some sense stateless.
More precisely, the Strong Diffie-Hellman oracle computes its answer to a query
in a way that is independent of the content of any previous queries.
Second, the ``winning condition'' in the Strong Diffie-Hellman game is essentially independent of
the queries made to the Strong Diffie-Hellman oracle.
\TODO{explain why this is the case.}
\TODO{give at least a vague reason for why this should allow the proofs to work.}

We can define a class of interactive computational assumptions which satisfy the above two criteria.
% This will allow us to prove impossibility results for this class.
We will call such assumptions \emph{simple computational assumptions}.

\begin{definition}[Simple computational assumption]\label{def:simple}
  A simple computational assumption is a tuple \((\Init,\Respond,\Win,\kappa(\cdot))\),
  where \(\kappa : \N \to \R\) is a function,
  \(\Init\) is a probabilistic algorithm
  and \(\Respond\) and \(\Win\) are deterministic algorithms.
  For a given simple computational assumption \(\Simple = (\Init,\Respond,\Win,\kappa)\)
  and an algorithm \advA, we define an experiment \(\SICA_{\Simple}^{\advA}\)
  according to Figure \ref{fig:simple}.

  \TODO{Explain in more detail what the different algorithms actually do.}
  Here, \Win{} defines the ``winning condition'' in the game.
  Notice that \Win{} depends only on the initial state \(\state_{\Simple}\)
  of the game and the challenge solution \(s\) output by \advA
  (in particular, it does not depend on the queries made by \advA
  to \oracle during the game).
  Also, whenever \advA queries the oracle \oracle,
  the response is computed deterministically as \(\Respond(\state_{\Simple},m)\).
  The advantage of \advA{} against \Simple is given by
  \TODO{advantage notation.}
  \begin{equation}
    \abs[\big]{\prob{\SICA(\advA,\lambda) \implies 1} - \kappa(\lambda)}.
  \end{equation}
  \TODO{say something about kappa.}
\end{definition}

\begin{figure}
  \begin{pchstack}[center,space=0.5cm]
    \proc{Game $\SICA_{\Simple}^{\advA}(\lambda)$}{
      \label{code:simple:init} (\state_{\Simple}, c) \assign \Init(1^{\lambda}).\\
      \label{code:simple:output} s \assign \advA^{\oracle}(c).\\
      \label{code:simple:win} b' \assign \Win(\state_{\Simple},s).\\
      \pcreturn b'.
    }
    \proc{Oracle \oracle(m)}{
      \pcreturn \Respond(\state_{\Simple},m).
    }
  \end{pchstack}\caption{}\label{fig:simple}
\end{figure}

\TODO{say something about how this extends non-interactive assumptions.}
\TODO{say something about how CCA-type games can be modeled by this. And stDH}.


%%% Local Variables:
%%% mode: latex
%%% TeX-master: "../main"
%%% End:
