
\section{Preliminaries}

\TODO{introduce notation and things in a short paragraph.}

\subsection{Key exchange mechanisms}

\TODO{Add definition of a KEM.}

\begin{definition}[\((1-\delta)\)-correctness of a KEM]\label{def:corr}
  Let \(\KEM = (\Setup, \KG, \Encaps, \Decaps)\) be a KEM
  and let \(\delta : \mathbb{N} \to \mathbb{R}\) be a function.
  Then \KEM is said to have \emph{\((1-\delta)\)-correctness} if
  \begin{equation}
    \prob{\Decaps{}(\sk,c) \ne k : \params \assign \Setup{}(1^{\lambda}), (\sk,\pk) \assign \KG{}(\params), (c,k) \assign \Encaps{}(\pk)} \le \delta (\lambda)
  \end{equation}
\end{definition}

\TODO{Include the definition of ECPA security from HLG21.
  Motivate why we use a one-way version of their game instead.}

\begin{definition}[OW-ECPA security]\label{def:OWECPA_sec}
  Let \(\KEM = (\Setup, \KG, \Encaps, \Decaps)\) be a KEM
  and let \advA{} be an adversary in the \OWECPA{} game against \KEM (see Figure \ref{fig:ecpa}).
  Then the advantage of \advA{} against \OWECPA in \KEM is given by
  \begin{equation}
    \abs*{\prob{\OWECPA{}_{\KEM}^{\advA}(\mu,\rho,\lambda) \implies 1} - 1/\abs{\keyspace}},
  \end{equation}
  where \(\keyspace = \keyspace{}(\params)\) is the encapsulation key space of \KEM.
  \TODO{advantage notation?}
\end{definition}

\begin{figure}
  \begin{pcvstack}[center,space=0.5cm]
    \proc{Game $\OWECPA_{\KEM}^{\advA}(\mu, \rho, \lambda)$}{
      L_{rev}, L_{corr} \assign \List.\\
      \params \assign \Setup{}(1^{\lambda}).\\
      \pcfor i \in [\mu]:\\
      \t (\sk_i,\pk_i) \assign \KG(\params).\\
      \oracle := \{\Enc, \Reveal, \Corrupt\}.\\
      (i^*, j^*, k^*) \assign \advA^{\oracle}(\params,\pk_1, \ldots, \pk_{\mu}).\\
      \pcif i^{*} \notin L_{corr} \land i^{*} \notin L_{rev} \land k^{*} = k_{i^{*},j^{*}}:\\
      \t \pcreturn 1.\\
      \pcelse:\\
      \t \pcreturn 0.
    }
    \proc{Oracle \Enc{}$(i)$}{
      \text{Assume this is the \(j\)-th query to \Enc{} on user \(i\) with } j \in [\rho].\\
      (c_{i,j}, k_{i,j}) \assign \Encaps{}(\pk_i).\\
      \pcreturn c_{i,j}.
    }
    \begin{pchstack}[space=1cm]
      \proc{Oracle \Reveal{}$(i,j)$}{
        L_{rev} \assign L_{rev} \cup \{(i,j)\}.\\
        \pcreturn k_{i,j}.
      }
      \proc{Oracle \Corrupt{}$(i)$}{
        L_{corr} \assign L_{corr} \cup \{i\}.\\
        \pcreturn \sk_i.
      }
    \end{pchstack}
  \end{pcvstack}\caption{}\label{fig:ecpa}
\end{figure}

\subsection{Security games}
\TODO{Explain what this is, informally.}
\TODO{Explain what we mean by a \emph{challenger} and an \emph{adversary}, which is terminology I use frequently.}

\subsection{Reductions}


\begin{definition}[Simple reduction]
  \TODO{add this.}
\end{definition}


We define the security loss of a reduction \redR similarly to ???,
as a maximum over all adversaries \advA of a certain product of the
advantages and running times of \(\redR\) and \(\advA\).

\begin{definition}[Security loss]\label{def:loss}
  Let \(\redR\) be a simple reduction
  from breaking problem \(P_{1}\) to breaking problem \(P_{2}\).
  Then the security loss \(\ell : \N \to \N\) of \(\redR\) is given by
  \begin{equation}
    \ell(\lambda) = \sup_{\advA} \left\{ \frac{\epsilon_{\advA}(\lambda)}{\epsilon_{\redR^{\advA}}(\lambda)}\frac{T_{\redR^{\advA}}(\lambda) + T_{\advA}(\lambda)}{T_{\advA}(\lambda)} \right\}
  \end{equation}
  where the maximum is taken over all adversaries \(\advA\) in \(P_{1}\),
  and \((\epsilon_{\advA}, T_{\advA})\) and \((\epsilon_{\redR^{\advA}}, T_{\redR^{\advA}})\)
  denote the advantage and running time of \advA and \(\redR^{\advA}\) in \(P_{1}\) and \(P_{2}\), respectively.
\end{definition}

\TODO{comment on TA + TR}.

\TODO{write.}

\subsection{Computational assumptions}

When we prove our tightness impossibility results,
we will only be able to rule out tight security
from a specific class of computational assumptions.
Phrased in terms of reductions,
we will only be able to show that a tight reduction
from breaking a certain computational problem \(P_1\)
to breaking another computational problem \(P_2\) is impossible,
if the problem \(P_2\) is sufficiently restricted.
\TODO{change the first sentence to refer to problems rather than assumptions?}
This is in some sense an inherent limitation,
as we could not rule out a tight reduction from breaking problem \(P_1\)
to breaking \emph{any} other computational problem \(P_2\)
(a standard counterexample to this is to take \(P_2 = P_1\).
Clearly there is a tight reduction from breaking \(P_1\) to breaking \(P_1\)).
In order to rule out tight reductions from breaking \(P_1\) to breaking \(P_2\),
we will need the problem \(P_2\) to be ``harder''
than the problem \(P_1\), in a certain sense.

In the impossibility results of \TODO{cite},
tight security is ruled out from the class of \emph{non-interactive}
computational assumptions.
These assumptions are modeled by experiments in which an adversary receives a challenge \(c\)
and must output a solution \(s\) to the challenge, with no further interaction between the adversary and the experiment.
Following an observation of \TODO{cite}, we will actually slightly expand the class of computational assumptions
from which we can rule out tight reductions
compared to \TODO{cite}, without the need for any essentially new techniques.
In \TODO{cite}, it is shown that a tight, simple reduction
from breaking the weak forward secrecy of a certain class of key exchange protocols
to breaking the \emph{Strong Diffie-Hellman} problem
cannot exist, unless the Strong Diffie-Hellman problem is itself easy.
The key point here is that the Strong Diffie-Hellman game is interactive,
and hence does not fit into the class of non-interactive assumptions considered in e.g \TODO{cite}.
Nevertheless \TODO{cite} are able to apply a similar meta-reduction strategy
to the one pioneered in \TODO{cite} to prove their impossibility result.
They observe that their proof carries over to many other
interactive games, so long as the ``winning condition''
in the game is independent of the sequence of oracle queries made by the adversary.

We expand upon the observation of \TODO{cite}
by noting that their proof relies on a second, crucial property of the Strong Diffie-Hellman problem:
namely that the Strong Diffie-Hellman oracle computes its
responses to the adversary's queries
independently of any of the adversary's previous oracle queries.
In other words, the Strong Diffie-Hellman oracle is in a sense ``stateless''.
Briefly summarized,
this property is needed to ensure that
the meta-reduction of \TODO{cite}
correctly simulates the Strong Diffie-Hellman oracle
towards the reduction, in spite of the fact
that the meta-reduction rewinds the reduction
and hence may have a different history of oracle queries
than the reduction itself.

We will call the class of computational assumptions
that satisfy the above-mentioned criteria
\emph{simple computational assumptions}.

% \TODO{explain why this is the case.}
% \TODO{give at least a vague reason for why this should allow the proofs to work.}

\begin{definition}[Simple computational assumption]\label{def:simple}
  A simple computational assumption is a tuple \((\Init,\Respond,\Win,\kappa(\cdot))\),
  where \(\kappa : \N \to \R\) is a function,
  \(\Init\) is a probabilistic algorithm
  and \(\Respond\) and \(\Win\) are deterministic algorithms.
  For a given simple computational assumption \(\Simple = (\Init,\Respond,\Win,\kappa)\)
  and an algorithm \advA, we define an experiment \(\SICA_{\Simple}^{\advA}\)
  according to Figure \ref{fig:simple}.

  \TODO{Explain in more detail what the different algorithms actually do.}
  Here, \Win{} defines the ``winning condition'' in the game.
  Notice that \Win{} depends only on the initial state \(\state_{\Simple}\)
  of the game and the challenge solution \(s\) output by \advA
  (in particular, it does not depend on the queries made by \advA
  to \oracle during the game).
  Also, whenever \advA queries the oracle \oracle,
  the response is computed deterministically as \(\Respond(\state_{\Simple},m)\).
  The advantage of \advA{} against \Simple is given by
  \TODO{advantage notation.}
  \begin{equation}
    \abs[\big]{\prob{\SICA(\advA,\lambda) \implies 1} - \kappa(\lambda)}.
  \end{equation}
  \TODO{say something about kappa.}
\end{definition}

\begin{figure}
  \begin{pchstack}[center,space=0.5cm]
    \proc{Game $\SICA_{\Simple}^{\advA}(\lambda)$}{
      \label{code:simple:init} (\state_{\Simple}, c) \assign \Init(1^{\lambda}).\\
      \label{code:simple:output} s \assign \advA^{\oracle}(c).\\
      \label{code:simple:win} b' \assign \Win(\state_{\Simple},s).\\
      \pcreturn b'.
    }
    \proc{Oracle \oracle(m)}{
      \pcreturn \Respond(\state_{\Simple},m).
    }
  \end{pchstack}\caption{}\label{fig:simple}
\end{figure}

\TODO{define what it means for S to be ``hard''.}

\begin{remark}
  We note that none of the algorithms \((\Init,\Respond,\Win)\)
  in our definition need to be efficient.
  In particular, the challenger in a \SICA
  game against a simple computational assumption
  need not be efficient,
  and hence the simple computational assumption need not be \emph{falsifiable},
  using the terminology of \TODO{cite}.
  The predicate \(\Win\) likewise need not be efficient,
  and hence the winning condition in a \SICA game
  need not be efficiently verifiable.
\end{remark}

\begin{remark}
  It is clear that the class of simple computational assumptions
  contains the class of non-interactive computational assumptions.

  \TODO{say something about how this extends non-interactive assumptions.
  Argue that it is \emph{strictly} stronger than non-interactive assumptions.}
  \TODO{say something about how CCA-type games can be modeled by this. And stDH}.
\end{remark}



%%% Local Variables:
%%% mode: latex
%%% TeX-master: "../main"
%%% End:
