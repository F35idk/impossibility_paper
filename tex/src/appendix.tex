
\section{Supporting material}

\subsection{Deriving Theorem \ref{thm:main} from Theorem \ref{thm:etpa}}

In this section, we show how to formally derive Theorem \ref{thm:main}
from Theorem \ref{thm:etpa}.

\begin{namedproof}{Proof of Theorem \ref{thm:main}}
  Let \redR be a simple reduction
  taking adversaries in the \((\mu,\rho)\)-\ETPA game against \TP to
  adversaries in the \SICA game against \Simple,
  where \(\mu : \N \to \N\) and \(\rho : \N \to \N\) are polynomially bounded.
  We need to lower bound the security loss \(\ell\) of \redR, as given in Definition \ref{def:loss}.
  We will show that
  \begin{equation}\label{eq:show}
    \ell \ge \frac{1}{2}(1 - 1 / \abs{\SO}) \min \left\{ \mu, \rho / (5q) \right\} - 1 - \negl,
  \end{equation}
  where \SO is the solution space of the trapdoor puzzle scheme \TP
  (which we may assume satisfies \(\abs{\SO} \ge 2\)).
  The equation \eqref{eq:main} will then follow.

  Using Theorem \ref{thm:etpa},
  we obtain an adversary \advA against the \((\mu,\rho)\)-\ETPA security of \TP
  and an adversary \redM against the \SICA security of \Simple
  satisfying \eqref{ineq:A} and \eqref{ineq:M}
  and where the running time of \redM is at most \eqref{ineq:M:time}.
  We need to select the parameters \(t\) and \(\alpha\) of Theorem \ref{thm:etpa}
  so as to be able to deduce \eqref{eq:show}.
  % We wish to select \(t\) and \(\alpha\)
  % so as to make the success probability of \advA in \eqref{ineq:A} close to 1,
  % whereas the right hand side of \eqref{ineq:M} should be close to \(1/\mu + 5q/\rho\).
  % We also need the running time \(T_{\redM}\) of \redM to be polynomial in \(\lambda\).
  We may assume that \(\rho \ge 10 q\), or else \eqref{eq:show} is trivial.
  Setting \(t = \rho / 5\), \(\alpha = q/t\) and noting that
  \(\alpha =  5 q / \rho \le 1/2\) by the above assumption, we find that
  \begin{align}
    4^{q} \cdot \left( \frac{t}{\rho - \alpha t} \right)^{\alpha t}
    = 4^{q} \cdot \left( \frac{1}{\rho / t - \alpha} \right)^{q}
    & = 4^{q} \cdot \left( \frac{1}{5 - \alpha} \right)^{q} \notag\\
    & \le 4^{q} \cdot \left( \frac{1}{5 - 1/2} \right)^{q} \notag\\
    & = \negl, \notag
  \end{align}
  where for the final line we use that we may assume that
  \(q(\lambda) = \omega(\log{\lambda})\)
  (indeed, if this were not the case, the brute-force attack of the adversary \advA
  given by Theorem \ref{thm:etpa}
  would be efficient in and of itself. In this case, we would have nothing to prove).
  Using this, we now get
  \begin{align}\label{ineq:eps}
    \abs{\epsilon_{\redR^{\advA}} - \epsilon_{\redM}} \le 1/\mu + \alpha + \negl,
  \end{align}
  along with
  \begin{align}
    p_{\advA} \ge 1 - 1/\mu - \delta \mu \rho - \negl
    = 1 - 1/\mu - \negl,
  \end{align}
  since the correctness error \(\delta\) of \TP satisfies \(\delta(\lambda) = \negl\) by assumption.
  It follows that the advantage \(\epsilon_{\advA}\) of \advA is
  \[
  \epsilon_{\advA} = \abs[\big]{p_{\advA} - 1/\abs{\keyspace}} \ge 1 - 1/\abs{\keyspace} - 1/\mu - \negl.
  \]

  On the other hand, by our definition of a trapdoor puzzle scheme (\TODO{ref.}),
  the algorithm \(\PuzzleSolve\) is efficient,
  and so its running time \(T_{P}\) is polynomially bounded.
  Hence the running time \(T_{\redM}\) of \redM is
  \[
    T_{\redM} \le (\mu + 1) \cdot T_{\redR^{\advA}} + (\mu \rho + 1) \cdot T_{P}
    = \poly.
  \]
  Thus \redM is an efficient adversary against the \SICA security of \Simple,
  and so if \Simple is hard then
  \(
    \epsilon_{\redM} = \negl.
  \)
  Hence \eqref{ineq:eps} now gives
  \begin{align}
    \epsilon_{\redR^{\advA}} \le  \epsilon_{\redM} + 1/\mu + \alpha + \negl \le 1/\mu + \alpha + \negl. \notag
  \end{align}
  We thus find that
  \begin{align}\label{ineq:insertinto}
    \frac{\epsilon_{\advA}}{\epsilon_{\redR^{\advA}}} \ge \frac{1 - 1 / \abs{\keyspace} - 1 / \mu - \negl}{1 / \mu + \alpha + \negl}
    & = \frac{1 - 1 / \abs{\keyspace}}{1 / \mu + \alpha + \negl}\\
    & - \frac{1/\mu + \negl}{1/\mu + \alpha + \negl}. \notag
  \end{align}
  We will bound the two terms on the right hand side of \eqref{ineq:insertinto}.
  For the first term, using \(1/\mu + \alpha \le 2 \max \{1/\mu, \alpha\}\)
  along with Taylor's theorem gives
  \begin{align}
    \frac{1}{1/\mu + \alpha + \negl}
    & \ge \frac{1}{2 \max \{1 /\mu, \alpha\} + \negl} \notag \\
    & \ge \frac{1}{2 \max \{1 /\mu, \alpha\}} - \negl \cdot \left(  \frac{1}{2 \max \{1 /\mu, \alpha\}} \right)^{2} \notag \\
    & = \frac{1}{2} \min \{ \mu, \alpha^{-1}\} - \negl, \notag
    % \mu - \mu^{2}(\alpha + \negl) & = \mu - \mu^{2} \alpha - \negl \notag\\
    % \frac{1}{1/\mu + \alpha + \negl} \ge \mu - \mu^{2}(\alpha + \negl) & = \mu - \mu^{2} \alpha - \negl \notag\\
    % & \ge \mu - \mu^{2} 5 q / \rho - \negl, \notag
  \end{align}
  and recalling that \(\alpha = 5q / \rho\) yields
  \begin{align}\label{ineq:insertthis1}
    \frac{1}{1/\mu + \alpha + \negl}
    & \ge \frac{1}{2} \min \{ \mu, \rho / (5q)\} - \negl.
  \end{align}
  Similarly, for the second term we find that
  \begin{align}\label{ineq:insertthis2}
    \frac{1/\mu + \negl}{1/\mu + \alpha + \negl} \le \frac{1/\mu + \negl}{1/\mu + \negl} = 1 + \negl.
  \end{align}
  Inserting \eqref{ineq:insertthis1} and \eqref{ineq:insertthis2} into \eqref{ineq:insertinto} gives
  \begin{align}
    \frac{\epsilon_{\advA}}{\epsilon_{\redR^{\advA}}} \ge \frac{1}{2}(1 - 1 / \abs{\keyspace}) \min \left\{ \mu, \rho / (5q) \right\} - 1 - \negl. \notag
    % \frac{\epsilon_{\advA}}{\epsilon_{\redR^{\advA}}} \ge (1 - 1/\abs{\keyspace})(\mu - 5 \mu^{2} q / \rho) - 1 - \negl. \notag
  \end{align}
  Using the definition of security loss (\ref{def:loss}), we now find that
  \[
    \ell \ge \frac{\epsilon_{\advA}}{\epsilon_{\redR^{\advA}}}\frac{T_{\redR^{\advA}} + T_{\advA}}{T_{\advA}}
    \ge \frac{\epsilon_{\advA}}{\epsilon_{\redR^{\advA}}}
    \ge \frac{1}{2}(1 - 1 / \abs{\keyspace}) \min \left\{ \mu, \rho / (5q) \right\} - 1 - \negl, \notag
  \]
  as required.
\end{namedproof}


%%% Local Variables:
%%% mode: latex
%%% TeX-master: "../main"
%%% End:
